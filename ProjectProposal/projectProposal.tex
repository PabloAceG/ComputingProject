\documentclass{article}

% Lists
\usepackage{listing}
% Bibliography
\usepackage[backend=biber, style=numeric]{biblatex}
\addbibresource{projectproposal.bib}

% Title
\title{Project Proposal}
\author{Pablo Acereda}
\date{10 Oct 2019}

% Document
\begin{document}

\maketitle

% %%%%%%%%%%%%%%%%%%%%%%%%%%%%%%%%%%%%%%%%%%%%%%%%%%%%%%%%%%%%%%%%%%%%%%%%%%%%%%
%                           DEFINING THE PROJECT
% %%%%%%%%%%%%%%%%%%%%%%%%%%%%%%%%%%%%%%%%%%%%%%%%%%%%%%%%%%%%%%%%%%%%%%%%%%%%%%

\section{Defining the Project}

%                            DETAILED RESEARCH
%                           ===================
\subsection{Detailed Research}

\textbf{Question:} How complexity metrics are correlated to the outcome of
supervise algorithms? Do complexity metrics affect the outcome of Feature
Selection Algorithms? How complexity metrics and imbalance are related? Do
complexity metrics tell us something about the quality of the datasets? 

\textbf{Aim:} To explore complexity metrics on software defect datasets.

\textbf{Objectives:} 

\begin{enumerate}

\item Literature search and literature review on complexity metrics, software
defects, prediction techniques, etc.

\item Develop suitable models for Ho and Basu complexity metrics.

\item Identify and collect suitable data for analysis and evaluation.

\item Complete final report.

\end{enumerate}


%                                KEYWORDS
%                               ==========
\subsection{Keywords}

Data Mining, Data Quality, Software Defect Prediction, Complexity Metrics, 
Imbalance.

%                              PROJECT TITLE
%                             ===============
\subsection{Project Title}

Analyse Complexity Metrics in Software Defect Prediction.

%                      CLIENT, AUDIENCE AND MOTIVATION
%                     =================================
\subsection{Client, Audience and Motivation}

The target group of this paper is mainly academics. Those who would be more
interested among that generalization, would be the academics in OBU and UAH; who
could benefit from the new knowledge generated about how complexity metrics
affect classification, attribute selection, etc.

%                     PROJECT PLAN AND RISK MANAGEMENT
%                    ==================================
\subsection{Project Plan and Risk Management}

\begin{enumerate}

\item Literature search - 1 week
\item Literature review - 3 weeks
\item Design and implementation - 2 weeks
\item Evaluation and Report Completion - 1 week

\end{enumerate}

% %%%%%%%%%%%%%%%%%%%%%%%%%%%%%%%%%%%%%%%%%%%%%%%%%%%%%%%%%%%%%%%%%%%%%%%%%%%%%%
%                          INITIAL LITERATURE REVIEW
% %%%%%%%%%%%%%%%%%%%%%%%%%%%%%%%%%%%%%%%%%%%%%%%%%%%%%%%%%%%%%%%%%%%%%%%%%%%%%%

\section{Initial Literature Review}

%                                 ABSTRACT
%                                ==========
\subsection{Abstract}

In this paper it is going to be analysed the complexity metrics affect
classification, attribute selection, imbalance, etc., using supervised 
classification models on publicly available datasets. The obtained results 
are to be compared with the metrics from the dataset; regarding the attribute 
selection.

To achieve more accurate results, cross-validation methods are also implemented.

The experiment is going to be done using Python as programming language.

%                        INITIAL LITERATURE REVIEW
%                       ===========================
\subsection{Initial Literature Review}

%                                RELEVANT
%                               ==========
\subsection{Relevant Professional, Social, Ethical, Security and Legal Issues to
the Project}

%                              BIBLIOGRAPHY
%                             ==============
\subsection{Bibliography}

\nocite{*}
\printbibliography

\end{document}

