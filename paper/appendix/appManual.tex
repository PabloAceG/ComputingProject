\chapter{Prerequisites}\label{chp:prerequisites}

Some relevant code sections have been included. The whole project is publicly 
available on GitHub: \url{https://github.com/PabloAceG/ComputingProject}.

%%%%%%%%%%%%%%%%%%%%%%%%%%%%%%%%%%%%%%%%%%%%%%%%%%%%%%%%%%%%%%%%%%%%%%%%%%%%%%%%

To be able to execute the experiments within the repository, Python3 is needed. 
Anaconda or the official Python located in the official repositories can be used 
as long as version 3 or posterior is used. Trying to replicate the experiments 
on some operative systems might terminate in error. If this is the case, python
can be changed (\textit{Linux}) with:

\begin{lstlisting}[language=bash]
    sudo update-alternatives --config python
\end{lstlisting}

R (programming language) is needed before trying to execute the project. Also, 
the following packages are mandatory in order to replicate the experiments:

\begin{itemize}
    \item ECoL - Dataset Complexity Metrics Package.
        \begin{itemize}
            \item \url{https://github.com/lpfgarcia/ECoL}
            \item \url{https://cran.r-project.org/web/packages/ECoL/}
        \end{itemize}
    \item Rserve - server, responds requests made to R: 
    \url{https://rforge.net/Rserve/doc.html}.
\end{itemize}

Once the previous requisites are fulfilled, the R server can be started by 
executing the following commands:

\begin{lstlisting}[language=bash]
library(Rserve) # import the library
run.Rserve() # start the server. Or simply Rserve()
\end{lstlisting}

Now, it is time to download the project to install the remaining Python 
packages. The project can be downloaded from 
\url{https://github.com/PabloAceG/ComputingProject/}.

Same as before, some Python packages are mandatory to execute the project. These
packages are available in requirements.txt
\footnote{\url{https://github.com/PabloAceG/ComputingProject/blob/master/code/requirements.txt}} 
file. To automatically install those packages, run\footnote{All commands are 
executed from the parent folder of the repository.}:

\begin{lstlisting}[language=bash]
pip install -r .\code\requirements.txt
\end{lstlisting}

It might happen that \lstinline{pip install -r} might not install all packages. 
To solve this, the failing packages must be installed manually:

\begin{lstlisting}[language=bash]
pip install <package_name>
\end{lstlisting}

Now, the experiments should be replicable. The experiment's code is under the 
folder \url{https://github.com/PabloAceG/ComputingProject/tree/master/code}
To run them, execute: 

\begin{lstlisting}[language=bash]
python code/metrics_comparison.py
python code/metrics_kfold.py
python code/metrics_kfold_undersampling.py
python code/metrics_kfold_oversampling.py
\end{lstlisting}

Each of the previous commands execute one experiment.

As final remarks, the class \lstinline{r_connect.py}
\footnote{\url{https://github.com/PabloAceG/ComputingProject/blob/master/code/r_connect.py}}
is the client connection the server in R (Rserve). It makes the requests to the 
\lstinline{ECoL} package to obtain the complexity metrics.

The class \lstinline{data.py}
\footnote{\url{https://github.com/PabloAceG/ComputingProject/blob/master/code/data.py}}
standardizes the datasets input (parsing data) and some other metrics from the 
package sklearn \footnote{\url{https://scikit-learn.org/stable/index.html}}.
