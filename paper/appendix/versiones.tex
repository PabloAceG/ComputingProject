%%%%%%%%%%%%%%%%%%%%%%%%%%%%%%%%%%%%%%%%%%%%%%%%%%%%%%%%%%%%%%%%%%%%%%%%%%%
%
% Generic template for TFC/TFM/TFG/Tesis
%
% $Id: herramientas.tex,v 1.5 2014/01/08 22:56:03 macias Exp $
%
% By:
%  + Javier Mac�as-Guarasa. 
%    Departamento de Electr�nica
%    Universidad de Alcal�
%  + Roberto Barra-Chicote. 
%    Departamento de Ingenier�a Electr�nica
%    Universidad Polit�cnica de Madrid   
% 
% Based on original sources by Roberto Barra, Manuel Oca�a, Jes�s Nuevo,
% Pedro Revenga, Fernando Herr�nz and Noelia Hern�ndez. Thanks a lot to
% all of them, and to the many anonymous contributors found (thanks to
% google) that provided help in setting all this up.
%
% See also the additionalContributors.txt file to check the name of
% additional contributors to this work.
%
% If you think you can add pieces of relevant/useful examples,
% improvements, please contact us at (macias@depeca.uah.es)
%
% Copyleft 2013
%
%%%%%%%%%%%%%%%%%%%%%%%%%%%%%%%%%%%%%%%%%%%%%%%%%%%%%%%%%%%%%%%%%%%%%%%%%%%

\chapter{Versiones}
\label{cha:versiones}

En este apartado incluyo el historial de cambios m�s relevantes de la
plantilla a lo largo del tiempo.

No empec� este ap�ndice hasta principios de 2015, con lo que se ha
perdido parte de la informaci�n de los cambios importantes que ha ido
sufriendo esta plantilla.


\begin{itemize}
\item Enero 2015:
  \begin{itemize}
  \item Solucionado el problema (gordo) de compilaci�n del
    \texttt{anteproyecto.tex} y el \texttt{book.tex}, debido al uso de
    paths distintos en la compilaci�n de la bibliograf�a. El sistema se ha
    complicado un poco (ver
    \texttt{biblio\textbackslash{}bibliography.tex}).
  \item A�adido un (rudimentario) sistema para generar pdf con las
    diferencias entre el documento en su estado actual y lo �ltimo
    disponible en el repositorio (usando \texttt{latexdiff}).
  \end{itemize}
\item Diciembre 2015:
  \begin{itemize}
  \item Separada la compilaci�n del anteproyecto de la del documento
    principal. Para el primero se ha creado el directorio
    \texttt{anteproyecto} donde est� todo lo necesario.
  \end{itemize}
\end{itemize}

%%% Local Variables:
%%% TeX-master: "../book"ve
%%% End:


