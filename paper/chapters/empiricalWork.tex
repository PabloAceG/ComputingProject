%%%%%%%%%%%%%%%%%%%%%%%%%%%%%%%%%%%%%%%%%%%%%%%%%%%%%%%%%%%%%%%%%%%%%%%%%%%
%
% Generic template for TFC/TFM/TFG/Tesis
%
% $Id: introduccion.tex,v 1.19 2015/02/24 23:21:54 macias Exp $
%
% By:
%  + Javier Macías-Guarasa. 
%    Departamento de Electrónica
%    Universidad de Alcalá
%  + Roberto Barra-Chicote. 
%    Departamento de Ingeniería Electrónica
%    Universidad Politécnica de Madrid   
% 
% Based on original sources by Roberto Barra, Manuel Ocaña, Jesús Nuevo,
% Pedro Revenga, Fernando Herránz and Noelia Hernández. Thanks a lot to
% all of them, and to the many anonymous contributors found (thanks to
% google) that provided help in setting all this up.
%
% See also the additionalContributors.txt file to check the name of
% additional contributors to this work.
%
% If you think you can add pieces of relevant/useful examples,
% improvements, please contact us at (macias@depeca.uah.es)
%
% Copyleft 2013
%
%%%%%%%%%%%%%%%%%%%%%%%%%%%%%%%%%%%%%%%%%%%%%%%%%%%%%%%%%%%%%%%%%%%%%%%%%%%

\chapter{Empirical Work}\label{chp:empwork}

In this section is about the experimental work carried out through this 
research. Firstly, describes the datasets used for the experimentation;
then, the supervised classifiers evaluated; the evaluation metrics chosen; and 
finally, present and discuss the results.

%%%%%%%%%%%%%%%%%%%%%%%%%%%%%%%%%%%%%%%%%%%%%%%%%%%%%%%%%%%%%%%%%%%%%%%%%%%%%%%%
\section{Datasets}\label{sec:datasets}

In this work, publicly available datasets are going to be used, in the domain 
of Software defect prediction: use of Jureczko and Madeyski dataset
\footnote{\url{http://snow.iiar.pwr.wroc.pl:8080/MetricsRepo/}} 
\cite{Jureczko2010, MadeyskiJ2015} and the Harman Search Base dataset
\cite{Harman2014ssbse}. 

From the first cluster of datasets, 15 open source projects are the ones chosen 
(a total of 8 are used for this project). The number of defects found in each 
class collected from the Software Management System (SCM) using a regular 
expression. The datasets are publicly available  (can be found in the PROMISE 
repository~\cite{promiserepo}). These datasets have been used in previous 
researches, \cite{Xu2018, Wang2016, Xia2016}, which allows comparing and 
analyzing the obtained results.

A sample can be considered as \textit{defective} when the number of defects 
inside the class is more than 0. Similarly, if the number of defects is indeed
0, then the class is \textit{non-defective}. This allows a binary 
classification, making easier handling results, operations and comparisons.

\begin{center}
\begin{longtable}{ | l | l | }
\caption{Defect Metrics Dataset Variables}\label{tab:dataJureczkoMetrics} \\

\hline
 \emph{Metric} & \emph{Description} \\
\hline 
\hline
\endfirsthead
\multicolumn{2}{c}{\tablename\ \thetable\ -- \textit{Continued from previous page}} \\
\hline
\emph{Metric} & \emph{Description} \\
\hline 
\hline
\endhead
\hline
\multicolumn{2}{r}{\textit{Continued on next page}}
\endfoot
\hline
\endlastfoot

WMC    &    Weighted methods per class \cite{Chidamber1994} \\
DIT    &    Depth of Inheritance Tree \cite{Chidamber1994} \\
NOC    &    Number of Children \cite{Chidamber1994} \\
CBO    &    Coupling between object classes \cite{Chidamber1994} \\
RFC    &    Response for a Class \cite{Chidamber1994} \\
LCOM   &    Lack of cohesion in methods \cite{Chidamber1994} \\
Ca     &    Afferent couplings \cite{Martin1994} \\
Ce     &    Efferent couplings \cite{Martin1994} \\
NPM    &    Number of Public Methods \cite{Bansiya2002} \\
LCOM3  &    Lack of cohesion in methods \cite{Henderson-Sellers1995} \\
LOC    &    Lines of Code \cite{Bansiya2002} \\
DAM    &    Data Access Metric \cite{Bansiya2002} \\
MOA    &    Measure of Aggregation \cite{Bansiya2002} \\
MFA    &    Measure of Functional Abstraction \cite{Bansiya2002} \\
CAM    &    Cohesion Among Methods of Class \cite{Bansiya2002} \\
IC     &    Inheritance Coupling \cite{Tang} \\
CBM    &    Coupling Between Methods \cite{Tang} \\
AMC    &    Average Method Complexity \cite{Tang} \\
MAX\textunderscore CC    &    Maximum McCabe's cyclomatic complexity \cite{McCabe1976} \\
AVG\textunderscore CC    &    Average McCabe's cyclomatic complexity \cite{McCabe1976} \\
\hline
\end{longtable}
\end{center}

There is another sole dataset obtained from the same source as the previous 
collection of datasets, with a very similar structure of data, the 
\textit{Apache} dataset. It has the same origin and objective as its 
predecessors.

On the second cluster of datasets, the data comes from a total of 8 different
Hadoop versions. Their original purpose is to train a search based fault 
prediction system. 

Similarly to Jureczko and Madeyski dataset \cite{Jureczko2010, MadeyskiJ2015},
it uses: (1) WMC, (2) DIT, (3), NOC, (4) CBO, (5) RFC, (6) LCOM, (7) NOM, and
(8) LOC; metrics to define each set.

Regarding more information on the content of the datasets the next table (see
Table~\ref{tab:metrics-description}) summarizes the number of samples on each 
dataset and further valuable information, such as the absolute and
relative number of deffects for those datasets.

\begin{center}
\begin{longtable}{ | c | c | c c c c | }
\caption{Description of the Datasets} \label{tab:metrics-description} \\

\hline
\emph{Project} & \emph{Version} & 
\emph{\#instances} & \emph{\#Non-Defective} & \emph{\#Defective} & \emph{\%Defective} \\
\hline 
\hline
\endfirsthead
\multicolumn{6}{c}{\tablename\ \thetable\ -- \textit{Continued from previous page}} \\
\hline
\emph{Project} & \emph{Version} & 
\emph{\#instances} & \emph{\#Non-Def} & \emph{\#Def} & \emph{\%Def} \\
\hline
\hline
\endhead
\hline
\multicolumn{6}{r}{\textit{Continued on next page}}
\endfoot
\hline
\endlastfoot

\multirow{5}{*}{ant} &  
    1.3 &   125 &   105 &    20 &   16.00 \\
&   1.4 &   178 &   138 &    40 &   22.47 \\ 
&   1.5 &   293 &   261 &    32 &   10.92 \\
&   1.6 &   351 &   259 &    92 &   26.21 \\
&   1.7 &   745 &   579 &   166 &   22.28 \\

\hline
\multirow{1}{*}{apache} &  
    -   &   191 &   107 &   84  &   43.98 \\

\hline
\multirow{4}{*}{camel} &   
    1.0 &   339 &   326 &    13 &    3.83 \\
&   1.2 &   608 &   392 &   216 &   35.52 \\
&   1.4 &   872 &   727 &   145 &   16.62 \\
&   1.6 &   965 &   777 &   188 &   19.48 \\

\hline
\multirow{8}{*}{hadoop} &
    0.1 &   141 &    91 &    50 &   35.60 \\
&   0.2 &   191 &   149 &    42 &   21.99 \\
&   0.3 &   211 &   158 &    53 &   25.12 \\
&   0.4 &   201 &   159 &    42 &   20.90 \\
&   0.5 &   217 &   180 &    37 &   17.05 \\
&   0.6 &   234 &   203 &    31 &   13.25 \\
&   0.7 &   250 &   202 &    48 &   19.20 \\
&   0.8 &   240 &   224 &    16 &    6.67 \\

\hline
\multirow{2}{*}{ivy} &   
    1.4 &   241 &   225 &    16 &    6.63 \\
&   2.0 &   352 &   312 &    40 &   11.36 \\

\hline
\multirow{5}{*}{jedit} & 
    3.2 &   272 &   182 &    90 &   33.08 \\
&   4.0 &   306 &   231 &    75 &   24.50 \\
&   4.1 &   312 &   233 &    79 &   25.32 \\
&   4.2 &   367 &   319 &    48 &   13.07 \\
&   4.3 &   492 &   481 &    11 &    2.23 \\

\hline
\multirow{3}{*}{log4j} & 
    1.0 &   135 &   101 &    34 &   25.18 \\
&   1.1 &   109 &    72 &    37 &   33.94 \\
&   1.2 &   205 &    16 &   189 &   92.19 \\

\hline
\multirow{3}{*}{synapse} &
    1.0 &   157 &   141 &    16 &   10.19 \\
&   1.1 &   222 &   162 &    60 &   27.02 \\
&   1.2 &   256 &   170 &    86 &   33.59 \\

\hline
\multirow{4}{*}{xalan} & 
    2.4 &   723 &   613 &   110 &   15.21 \\
&   2.5 &   803 &   416 &   387 &   48.19 \\
&   2.6 &   885 &   474 &   411 &   46.44 \\
&   2.7 &   909 &    11 &   898 &   98.78 \\

\hline
\multirow{3}{*}{xerces} &
    1.2 &   440 &   369 &    71 &   16.13 \\
&   1.3 &   453 &   384 &    69 &   15.23 \\
&   1.4 &   588 &   151 &   437 &   74.31 \\
\hline

\end{longtable}
\end{center}

The experiments of this paper use the latest version of the datasets 
\textit{ant}, \textit{apache}, \textit{camel}, \textit{hadoop}, \textit{ivy},
\textit{jedit}, \textit{log4j}, \textit{xalan} and \textit{xerces}; alongide
all the available versions of the \textit{hadoop} dataset.

%%%%%%%%%%%%%%%%%%%%%%%%%%%%%%%%%%%%%%%%%%%%%%%%%%%%%%%%%%%%%%%%%%%%%%%%%%%%%%%%
\section{Supervised Classifiers}

This work makes use of several supervised learning algorithms. The 
experiments carried out use the following algorithms:

\begin{description}
    \item [Naive Bayes] (NB)~\cite{Mit97} is a classifier that works on 
    conditional probabilities, uses the Bayes theorem to predict the class for 
    each data input. Calculates the probability of a certain event, given prior 
    knowledge. This classifier assigns a set of attributes 
    $a_{1}, \ldots, a_{n}$ to a given class $C$ so that the probability of the 
    class description value of the attributes instances is maximal: 
    $P(C|a_{1}, \ldots, a_n)$. The probability of the hypothesis, given that 
    the evidence is true, is the probability of the evidence, given the 
    hypothesis is true, multiplied by the probability of the hypothesis; in 
    relation to the probability of the evidence. See Eq.~\ref{eq:bayes}.
    
    \begin{equation}\label{eq:bayes}
        P(H|E) = \frac{(E|H) * P(H)}{P(E)}
    \end{equation}
    
    For this project it has been selected the Gaussian Naive Bayes to perform
    the experimentation. Being \textit{Gaussian} means that the likelihood of
    the features is assumed to be Gaussian. The formula is given by 
    Eq.~\ref{eq:gaussian}.
    
    \begin{equation}\label{eq:gaussian}
        P(x_{i}|y)=(\frac{1}{\sqrt{2\pi \sigma^{2}_{y}}})\exp{(-\frac{(x_{i} - \mu _{y})^2}{2\sigma^{2}_{y}})}
    \end{equation}
    
    \item [CART] (Classification And Regression Trees)~\cite{Breiman1984} is a 
    non-parametric decision tree, similar to \textbf{C4.5}~\cite{Quinlan1993}. 
    The main difference is that this algorithm supports numerical target 
    variables. In exchange, it cannot compute rule sets. CART constructs a two 
    branch bifurcation of the most discriminating attribute, based on the
    Gini index. It can generate either classification or regression trees - 
    depends on the variable (categorical or numeric, respectively). 
    \textit{Classification and Regression Trees} algorithm is more complex and 
    time consuming than \textit{C4.5}'s since multiple trees need to be built 
    and pruned, but trees are generally simpler~\cite{oates1997}.
    
    The implementation available in ~\cite{scikit-learn} is an optimised 
    version of this algorithm, although it does not support categorical 
    variables.
    
    \item [Nearest Centroid] classifier\cite{conformal2014}, or \textbf{Nearest 
    Prototype} classifier, or \textbf{Rocchio classifier} for its similarity to 
    an algorithm with the same name (see Eq.\ref{eq:rocchio}). 
    
    \begin{equation}\label{eq:rocchio}
        \overrightarrow{Q_{m}} =  (a \cdot \overrightarrow{Q_{o}}) + (b \cdot 
        \frac{1}{\lvert D_{r} \rvert} \cdot \sum_{\overrightarrow{D_{j}} \in 
        D_{r}} \overrightarrow{D_{j}}) - (c \cdot \frac{1}{\lvert D_{nr} 
        \rvert} \cdot \sum_{\overrightarrow{D_{k}} \in D_{nr}} 
        \overrightarrow{D_{k}})
    \end{equation}
    
    The labels of a given sample are assigned by evaluating the classes of 
    training samples whose mean is closest to the evaluated point. The training 
    procedure, given a labeled training set ${(\overrightarrow{x_{1}}, y_{1}), 
    \ldots,  (\overrightarrow{x_{n}}, y_{n})}$ with class labels $y_{i} \in Y$, 
    compute the per-class centroid with Eq.~\ref{eq:knntrain}.
    
    \begin{equation}\label{eq:knntrain}
        \overrightarrow{\mu _{l}} = \frac{1}{\lvert C_{l} \rvert} \sum_{i \in 
        C_{l}} \overrightarrow{x_{i}}
    \end{equation}
    
    Where $C_{l}$ is the set of indices of samples belonging to class $l \in Y$.
    The prediction functions, takes the class assigned to an observation 
    $\overrightarrow{x}$ is Eq.~\ref{eq:knnpred}.
    
    \begin{equation}\label{eq:knnpred}
        \overrightarrow{y} = argmin_{l \in Y} \lvert\lvert \overrightarrow{
        \mu _{l}} - \overrightarrow{x} \rvert\rvert
    \end{equation}

\end{description}

%Classifiers can be adapted to deal with  imbalance in different ways.
%In this work, we have used the following base learners implemented in Weka:
% C4.5 (implemented as J48) - weights yes
% JRip weights - yes
% Naive Bayes - weithgts yes
% IB weights yes
% logistic regression weights yes
% NN weights yes
% simpleCART -no
% SMO weights - yes
%

%%%%%%%%%%%%%%%%%%%%%%%%%%%%%%%%%%%%%%%%%%%%%%%%%%%%%%%%%%%%%%%%%%%%%%%%%%%%%%%%
\section{Evaluation Metrics}\label{sec:ev-metrics}

Part of the process of applying a classification algorithm is to measure the 
success and the results of the training. In order to do it, some rates can be 
calculated out of the testing to the trained classifier. Here is where the 
Confusion Matrix comes at hand.

The Confusion Matrix (see Table~\ref{tab:conf-matrix}) allows us to summarize 
the performance of a given classification algorithm, it is also the foundation 
of many of the performance metrics used in classification by:

% --> Confusion Matrix
\begin{table}[h!]
\centering
\footnotesize
\caption{Confusion Matrix for Binary Classification}
\label{tab:conf-matrix}
\begin{tabular}{c c | p{2.5cm} | p{2.5cm} }
% ROW1
  & & \multicolumn{2}{c}{\emph{Actual Class}}\\
% ROW2
  & &    \emph{Positive} & \emph{Negative} \\
% ROW3
  \hline
  \multirow{4}{*}{\emph{Predicted Class}}
  &\emph{Positive} & True Positive \newline ($TP$) & False Positive \newline ($FP$)\\
% ROW 4
  %\hline
  \cline{2-4}
  &\emph{Negative} & False Negative \newline ($FN$) & True Negative\newline ($TN$)\\
\end{tabular}
\end{table}

\begin{description}
 \item [True Positive (TP)] - Data is correctly classified as positive.
 \item [True Negative (TN)] - Data is correctly classified as negative.
 \item [False Positive (FP)] - Data being negative classified as positive.
 \item [False Negative (FN)] - Data being positive classified as negative. 
\end{description} 

These specifications indicate that the input data should have a binary target: 
one value that can be classified as \textit{positive} and a second value that
can be classified as negative. From this statistical classification
\footnote{Also know as error matrix.}, many performance measures can be 
calculated.

Some of the most widely used metrics, and the ones calculated in this paper are 
the ones explained next.

%%%%%%%%%%%%%%%%%%%%%%%%%%%%%%%%%%%%%%%%%%%%%%%%%%%%%%%%%%%%%%%%%%%%%%
\subsection{Precision}    
 
Also know as Positive Predictive Value (PPV). It is the relation 
between the \textit{true positives} calculated and the overall positives
detected by the classification algorithm (see Eq.~\ref{eq:ppv}).
 
\begin{equation}\label{eq:ppv}
    PPV = \frac{TP}{TP + FP} = 1 - FDR
\end{equation}

%%%%%%%%%%%%%%%%%%%%%%%%%%%%%%%%%%%%%%%%%%%%%%%%%%%%%%%%%%%%%%%%%%%%%%
\subsection{Recall} 

Also known as sensitivity, hit rate, or True Positive 
Rate (TPR). It stands for the relation between the \textit{true positives}
calculated and the real number of positives (see Eq.~\ref{eq:tpr}).

\begin{equation}\label{eq:tpr}
    TPR = \frac{TP}{P} = \frac{TP}{TP + FN} = 1 - FNR
\end{equation}

%%%%%%%%%%%%%%%%%%%%%%%%%%%%%%%%%%%%%%%%%%%%%%%%%%%%%%%%%%%%%%%%%%%%%%
\subsection{Fall-out}

Also called False Positive Rate (FPR). It is the probability of rejecting 
(falsely) the null hypothesis\footnote{General statement or default position
that there is no relationship between two measured phenomena or no
association among groups.} for a particular test (see Eq.~\ref{eq:fpr}).

\begin{equation}\label{eq:fpr}
    FPR = \frac{FP}{N} = \frac{FP}{FP + TN} = 1 - TNR
\end{equation}
    
%%%%%%%%%%%%%%%%%%%%%%%%%%%%%%%%%%%%%%%%%%%%%%%%%%%%%%%%%%%%%%%%%%%%%%
\subsection{Balance Accuracy}

It goes by the acronym BA. It is a metric generally used to evaluate how
good is a (binary) classifier. It is a measure that comes in specially 
handy for imbalanced datasets. Its formula is represented by the mean of
\textit{sensitivity} and \textit{specificity} (see Eq.~\ref{eq:ba}).

\begin{equation}\label{eq:ba}
    BA = \frac{TPR + TNR}{2}
\end{equation}
    
%%%%%%%%%%%%%%%%%%%%%%%%%%%%%%%%%%%%%%%%%%%%%%%%%%%%%%%%%%%%%%%%%%%%%%
\subsection{F-Measure} 
 
Also known as $F_1$, or \textit{S\o rensen-Dice coefficient} (independently
developed by Thorvald S\o rensen~\cite{sorensen1948} and Lee Raymond 
Dice~\cite{dice1945}). It is the harmonic mean of precision and sensitivity, 
the two previous measures, which measures accuracy (see Eq.~\ref{eq:f1}). It is 
twice the relation of the multiplication between \textit{recall} and 
\textit{precision} and their addition. It is commonly used in highly imbalanced 
datasets. There are some criticism rewarding this measure, as it does not take 
into account the \textit{True Negative} (\textit{TN}) cases.

\begin{equation}\label{eq:f1}
    F_1 = 2 \cdot \frac{PPV \cdot TPR}{PPV + TPR} = \frac{2 \cdot TP}{2 \cdot TP + FP + FN}
\end{equation}
    
%%%%%%%%%%%%%%%%%%%%%%%%%%%%%%%%%%%%%%%%%%%%%%%%%%%%%%%%%%%%%%%%%%%%%%    
\subsection{MCC} 

The Matthews Correlation Coefficient (MCC)~\cite{Matthews1975} or \textit{phi} 
coefficient, a performance metric (see Eq.~\ref{eq:mcc}) that measures the 
quality of a binary classification robust to the imbalance problem. Its range 
values are  between -1 and +1, where -1 represents complete inconsistency 
(disagreement), 0 indicates that the prediction is no better than a random 
prediction; and +1 would be a perfect prediction.  

\begin{equation}\label{eq:mcc}
    MCC = \frac{TP \cdot TN - FP \cdot FN}{\sqrt{(TP + FP)(TP + FN)(TN + FP)(TN + FN)}}
\end{equation}

Other measures cannot be used when data is highly imbalanced, like 
\textit{accuracy} (\textit{Acc}) (defined by Eq.~\ref{eq:accuracy}), as it does 
not take into account the number of labels of different classes.

\begin{equation}\label{eq:accuracy}
    Acc = \frac{TP + TN}{TP + TN + FP + FN}
\end{equation}

%%%%%%%%%%%%%%%%%%%%%%%%%%%%%%%%%%%%%%%%%%%%%%%%%%%%%%%%%%%%%%%%%%%%%%
\subsection{Receiver Operating Characteristic Curve}\label{sec:roc}

The Receiver Operating Characteristic (ROC)\cite{Fawcett2006} Curve represents 
graphically the True Positive Rate (TPR) versus the False Positive Rate (FPR) 
as shown in Figure~\ref{fig:roc}.

\begin{figure}
\centering
\includegraphics[width=.4\textwidth]{roc} % width=9cm, height=6cm
\caption{ROC Curve}
\label{fig:roc}
\end{figure}

Once the curve is plotted, the more the curve gets similar to a slope of
$TPR = FPR$ the more imbalance that can be found in the dataset. It provides
graphical visualization of the results. 

The Area Under the ROC Curve (AUC) is a quality measure between positive and 
negative rates with a single value. This metric allows to compare models.

An approximation to the function can be calculated with Eq~\ref{eq:auc}.

\begin{equation}\label{eq:auc}
    AUC = \frac{1 + TP_{r} - FP_{r}}{2}
\end{equation}

Three unbiased metrics, i.e., AUC, F-score (\cite{sorensen1948}, 
\cite{dice1945}), Matthews Correlation Coefficient (MCC - \cite{Matthews1975}), 
are used to evaluate the performance on imbalanced datasets.

%%%%%%%%%%%%%%%%%%%%%%%%%%%%%%%%%%%%%%%%%%%%%%%%%%%%%%%%%%%%%%%%%%%%%%%%%%%%%%%%
\section{Results and discussion}

%%%%%%%%%%%%%%%%%%%%%%%%%%%%%%%%%%%%%%%%%%%%%%%%%%%%%%%%%%%%%%%%%%%%%%

\subsection{Methodology}

To answer the RQ in Section \ref{sec:aim-obj}, we run several experiments.
First, we analyze the complexity metrics (see Section~\ref{sec:ecol}) from all 
selected datasets (see Section~\ref{sec:datasets}). Then, we retrieve some
analytical metrics (see Section~\ref{sec:ev-metrics}), applying K-fold Cross
Validation to those datasets. Finally, we repeat the process applying some 
under/oversampling techniques to the K-Fold process (see 
Sections~\ref{sec:undersampling} and \ref{sec:oversampling}, respectively).

As the final objective is to see how complexity metrics affect classification -
and therefore the analytical metrics, a final section is going to summarize
the results obtained and the conclusions drawn out of those measures and answer
Research Questions 1 to 4.

%%%%%%%%%%%%%%%%%%%%%%%%%%%%%%%%%%%%%%%%%%%%%%%%%%%%%%%%%%%%%%%%%%%%%%
\subsection{Metrics Analysis}

As it has been mentioned in \textit{Data Complexity Metrics} (see 
Section ~\ref{sec:ecol}), the software package to calculate them is 
available in R, ECoL\footnote{\url{https://github.com/lpfgarcia/ECoL/}}. 
Therefore, it was  necessary to call it from the Python environment. 
In order to do so, the package RServe, a client server implementation 
for R workspace to execute functions from other environments like 
Python, was used. In other words, this is a connector that allow us 
to execute the ECoL complexity metrics functions.

The code regarding this script can be found in the Appendix 
\ref{chp:pythoncode}, section \ref{sec:rconnect}.

Regarding the experiment's content and objective, it has been 
implemented an script that requests the complexity metrics of a 
dataset and then displays its values on graphs comparing the results. 

The structure of the script tries to find and compare the complexity 
metrics obtained from different datasets, to see if there is a 
certain relation between metrics.

There has been create a script (see simplification of the algorithm in
Algorithm~\ref{alg:complexmetrics}) that connects to ECoL library in R
and returns the complexity metrics of a certain dataset (see full 
implementation of the script in Appendix~\ref{chp:pythoncode}, 
Section~\ref{sec:exp-kfold-code}).

\begin{breakablealgorithm}
    \caption{Datasets Complexity Metrics Comparison}
    \footnotesize
    \label{alg:complexmetrics}
    \begin{algorithmic}[1]
        \Require $\mathcal{D}$ dataset
        \Ensure $\mathcal{C}$ complexity metrics
        \State con $\leftarrow$ connectEcol()
        
        \ForAll {set in $\mathcal{D}$}
        	\State inputs, targets $\leftarrow$ getDataset(set)
        	
        	\State metrics $\leftarrow$ con.getMetrics(inputs, targets)
        	
        	\State $\mathcal{C}$.add(metrics)
        \EndFor
        
    \State plotComparison($\mathcal{C}$)
    \end{algorithmic}
\end{breakablealgorithm}

The results are also stored in CSV files, which can be represented
as tables \ref{tab:complMetricsWholeDatasets_PROMISE} and
\ref{tab:complMetricsWholeDatasets_PROMISE2}.

The experiment has been repeated for two independents sets of 
datasets (see Section~\ref{sec:datasets} for more details regarding
the datasets and their content). The first experiment's results are
summarized in the following table (see 
table~\ref{tab:complMetricsWholeDatasets_PROMISE}), whereas the second
experiment, regarding the Hadoop datasets is represented in 
table~\ref{tab:complMetricsWholeDatasets_PROMISE2}.

The obtained results are showed in the following figures: 
(\ref{fig:balance}) balance, (\ref{fig:correlation}) correlation, 
(\ref{fig:dimensionality}) dimensionality, (\ref{fig:linearity}) linearity,
(\ref{fig:neighborhood}) neighborhood, (\ref{fig:network}) network,
(\ref{fig:overlap}) overlap, and (\ref{fig:smoothness}) smoothness.

%%%%%%%%%%
\begin{figure}[h!]
    \centering
    \begin{subfigure}{0.496\textwidth}
        \includegraphics[width=0.99\linewidth]{figures/balance-C1.png}
        \caption{Balance C1}
        \label{fig:balance-c1}
    \end{subfigure}
    \begin{subfigure}{0.496\textwidth}
        \includegraphics[width=0.99\linewidth]{figures/balance-C2.png}
        \caption{Balance C2}
        \label{fig:balance-c2}
    \end{subfigure}
    \caption{Balance Measures}
    \label{fig:balance}
\end{figure}

%%%%%%%%%%
\begin{figure}[h!]
    \centering
    \begin{subfigure}{0.496\textwidth}
        \includegraphics[width=0.99\textwidth]{figures/correlation-C2.png}
        \caption{Correlation C2}
        \label{fig:correlation-c2}
    \end{subfigure}
    \begin{subfigure}{0.496\textwidth}
        \includegraphics[width=0.99\textwidth]{figures/correlation-C3.png}
        \caption{Correlation C3}
        \label{fig:correlation-c3}
    \end{subfigure}
\end{figure}
\begin{figure}[h!]\ContinuedFloat
    \centering
    \begin{subfigure}{0.496\textwidth}
        \includegraphics[width=0.99\textwidth]{figures/correlation-C4.png}
        \caption{Correlation C4}
        \label{fig:correlation-c4}
    \end{subfigure}
    \caption{Correlation Measures}
    \label{fig:correlation}
\end{figure}

%%%%%%%%%%
\begin{figure}[h!]
    \centering
    \begin{subfigure}{0.496\textwidth}
        \includegraphics[width=0.99\textwidth]{figures/dimensionality-T2.png}
        \caption{Dimensionality T2}
        \label{fig:dimensionality-t2}
    \end{subfigure}
    \begin{subfigure}{0.496\textwidth}
        \includegraphics[width=0.99\textwidth]{figures/dimensionality-T3.png}
        \caption{Dimensionality T3}
        \label{fig:dimensionality-t3}
    \end{subfigure}
    \caption{Dimensionality Measures}
    \label{fig:dimensionality}
\end{figure}

%%%%%%%%%%
\begin{figure}[h!]
    \centering
    \begin{subfigure}{0.496\textwidth}
        \includegraphics[width=0.99\textwidth]{figures/linearity-L1.png}
        \caption{Linearity L1}
        \label{fig:linearity-l1}
    \end{subfigure}
    \begin{subfigure}{0.496\textwidth}
        \includegraphics[width=0.99\textwidth]{figures/linearity-L2.png}
        \caption{Linearity L2}
        \label{fig:linearity-l2}
    \end{subfigure}
\end{figure}
\begin{figure}[h!]\ContinuedFloat
    \centering
    \begin{subfigure}{0.496\textwidth}
        \includegraphics[width=0.99\textwidth]{figures/linearity-L3.png}
        \caption{Linearity L3}
        \label{fig:linearity-l3}
    \end{subfigure}
    \caption{Linearity Measures}
    \label{fig:linearity}
\end{figure}

%%%%%%%%%%
\begin{figure}[h!]
    \centering
    \begin{subfigure}{0.496\textwidth}
        \includegraphics[width=0.99\textwidth]{figures/neighborhood-N1.png}
        \caption{Neighborhood N1}
        \label{fig:neighborhood-n1}
    \end{subfigure}
    \begin{subfigure}{0.496\textwidth}
        \includegraphics[width=0.99\textwidth]{figures/neighborhood-N2.png}
        \caption{Neighborhood N2}
        \label{fig:neighborhood-n2}
    \end{subfigure}
\end{figure}
\begin{figure}[h!]\ContinuedFloat
    \centering
    \begin{subfigure}{0.496\textwidth}
        \includegraphics[width=0.99\textwidth]{figures/neighborhood-N3.png}
        \caption{Neighborhood N3}
        \label{fig:neighborhood-n3}
    \end{subfigure}
    \begin{subfigure}{0.496\textwidth}
        \includegraphics[width=0.99\textwidth]{figures/neighborhood-N4.png}
        \caption{Neighborhood N4}
        \label{fig:neighborhood-n4}
    \end{subfigure}
\end{figure}
\begin{figure}[h!]\ContinuedFloat
    \centering
    \begin{subfigure}{0.496\textwidth}
        \includegraphics[width=0.99\textwidth]{figures/neighborhood-T1.png}
        \caption{Neighborhood T1}
        \label{fig:neighborhood-t1}
    \end{subfigure}
    \begin{subfigure}{0.496\textwidth}
        \includegraphics[width=0.99\textwidth]{figures/neighborhood-LSC.png}
        \caption{Neighborhood LSC}
        \label{fig:neighborhood-lsc}
    \end{subfigure}
    \caption{Neighborhood Measures}
    \label{fig:neighborhood}
\end{figure}

%%%%%%%%%%
\begin{figure}[h!]
    \centering
    \begin{subfigure}{0.496\textwidth}
        \includegraphics[width=0.99\textwidth]{figures/network-Density.png}
        \caption{Network Density}
        \label{fig:network-density}
    \end{subfigure}
    \begin{subfigure}{0.496\textwidth}
        \includegraphics[width=0.99\textwidth]{figures/network-ClsCoef.png}
        \caption{Network ClsCoef}
        \label{fig:network-clscoef}
    \end{subfigure}
\end{figure}
\begin{figure}[h!]\ContinuedFloat
    \centering
    \begin{subfigure}{0.496\textwidth}
        \includegraphics[width=0.99\textwidth]{figures/network-Hubs.png}
        \caption{Network Hubs}
        \label{fig:network-hubs}
    \end{subfigure}
    \caption{Network Measures}
    \label{fig:network}
\end{figure}

%%%%%%%%%%
\begin{figure}[h!]
    \centering
    \begin{subfigure}{0.496\textwidth}
        \includegraphics[width=0.99\textwidth]{figures/overlap-F1.png}
        \caption{Overlap F1}
        \label{fig:overlap-f1}
    \end{subfigure}
    \begin{subfigure}{0.496\textwidth}
        \includegraphics[width=0.99\textwidth]{figures/overlap-F1v.png}
        \caption{Overlap F1v}
        \label{fig:overlap-f1v}
    \end{subfigure}
\end{figure}
\begin{figure}[h!]\ContinuedFloat
    \centering
    \begin{subfigure}{0.496\textwidth}
        \includegraphics[width=0.99\textwidth]{figures/overlap-F2.png}
        \caption{Overlap F2}
        \label{fig:overlap-f2}
    \end{subfigure}
    \begin{subfigure}{0.496\textwidth}
        \includegraphics[width=0.99\textwidth]{figures/overlap-F3.png}
        \caption{Overlap F3}
        \label{fig:overlap-f3}
    \end{subfigure}
\end{figure}
\begin{figure}[h!]\ContinuedFloat
    \centering
    \begin{subfigure}{0.496\textwidth}
        \includegraphics[width=0.99\textwidth]{figures/overlap-F4.png}
        \caption{Overlap F4}
        \label{fig:overlap-f4}
    \end{subfigure}
    \caption{Overlap Measures}
    \label{fig:overlap}
\end{figure}

%%%%%%%%%%
\begin{figure}[h!]
    \centering
    \begin{subfigure}{0.496\textwidth}
        \includegraphics[width=0.99\textwidth]{figures/smoothness-S1.png}
        \caption{Smoothness S1}
        \label{fig:smoothness-s1}
    \end{subfigure}
    \begin{subfigure}{0.496\textwidth}
        \includegraphics[width=0.99\textwidth]{figures/smoothness-S2.png}
        \caption{Smoothness S2}
        \label{fig:smoothness-s2}
    \end{subfigure}
\end{figure}
\begin{figure}[h!]\ContinuedFloat
    \centering
    \begin{subfigure}{0.496\textwidth}
        \includegraphics[width=0.99\textwidth]{figures/smoothness-S3.png}
        \caption{Smoothness S3}
        \label{fig:smoothness-s3}
    \end{subfigure}
    \begin{subfigure}{0.496\textwidth}
        \includegraphics[width=0.99\textwidth]{figures/smoothness-S4.png}
        \caption{Smoothness S4}
        \label{fig:smoothness-s4}
    \end{subfigure}
    \caption{Smoothness Measures}
    \label{fig:smoothness}
\end{figure}

This way it is easier to compare results through datasets and metrics. To see 
the whole extent of the numerical values, see the tables at 
Appendix~\ref{chp:tables}. The data is summarized in tables \ref{sec:compl-oo} 
and \ref{sec:compl-hadoop}. To see the images in more detail (bigger) go to 
Appendix~\ref{chp:figures}, Section~\ref{sec:fig-compl-metrics}.

Further results are compared and discussed in Section~\ref{sec:compresults}.

\newpage

%%%%%%%%%%%%%%%%%%%%%%%%%%%%%%%%%%%%%%%%%%%%%%%%%%%%%%%%%%%%%%%%%%%%%%
\subsection{K-fold on Metrics Analysis}\label{sec:exp-kfold}

This experiment tries to find the impact of applying the K-fold 
Cross-Validation. It is a widely extended technique in data science that brings 
a solution to the problem of performing testing to a data set with no separate
training data - a given data set should not be used for both training and 
testing for the classifier, instead data can be divided so a portion is used
for training and another separate portion for testing/validation. Cross 
validation allows to estimate the performance of the model trained by the whole 
dataset dividing it into folds of two types:

\begin{description}
    \item [Training set] Known data by the classifier, and used to train it. It 
    represents $k - 1 / k$ parts of the original set.
    \item [Testing set] Also known as validation set. Unknown data for the 
    algorithm. Used to test the performance of the trained classifier. It 
    usually is $1 / k$ of the original dataset.
\end{description}

The K-fold algorithm has a total of $k$ iterations, same as the number of folds 
(parts/divisions) of the dataset. On each iteration, the \textit{testing 
set} is rotated to another unused fold\footnote{The folds do not change 
throughout the cross validation process.} of the data - no testing set is 
repeated, which means that each fold is used once for \textit{testing} 
purposes. 

After all iterations, the performance of the classifier that would be obtained 
by training with the whole dataset is estimated by doing the mean of the
performances on each iteration of the algorithm. 

The most common values for $k$ (folds) used in data science are 5 and 10. This 
this experiment, $k$ has been selected as $k=5$, as a common value of $k$ was 
needed for all experiments and some of the datasets did not have enough 
samples for a larger value of $k$.

The algorithm that analyzes the performance of each fold and the calculates 
the mean performance can be simplified as the one explained in the 
Algorithm~\ref{alg:kfold}.

\begin{breakablealgorithm}
    \caption{Metrics Analysis on K-fold}
    \footnotesize
    \label{alg:kfold}
    \begin{algorithmic}[1]
        \Require $\mathcal{D}$ dataset
        \State inputs, targets $\leftarrow$ $\mathcal{D}$
    
        \State k $\leftarrow$ 5
        \State kf $\leftarrow$ kfold(k)
        \State splits $\leftarrow$ kf.split(inputs)
        
        \ForAll {$train_{i}$, $test_{i}$ in splits}
        	\State $x_{train}$, $x_{test}$ $\leftarrow$ inputs.get($train_{i}$), inputs.get($test_{i}$)
        	\State $y_{train}$, $y_{test}$ $\leftarrow$ targets.get($train_{i}$), targets.get($test_{i}$)
        	
        	\State clf $\leftarrow$ trainNetwork($x_{train}$, $y_{train}$)
        	
        	\State confusionMatrix($x_{test}$, $y_{test}$, clf)
        \EndFor
        \State calculateMeanPerformance(clf)
    \end{algorithmic}
\end{breakablealgorithm}

The real script used for this experiment can be found in 
Appendix~\ref{chp:pythoncode}, Section~\ref{sec:exp-kfold-code}. It basically
separates a certain dataset into folds and calculates the analysis metrics
for each fold, to later on compute the mean value of those metrics for the
entire dataset.

Thanks to the results of the results of the classifiers for each fold, first,
the \textit{confusion matrix} is obtained and thereafter, some other metrics 
are calculated (such as recall, MCC, etc.). Those metrics are then compared 
using a table like tab~\ref{tab:kfold} or \ref{tab:kfold2}, for each specific
dataset, to see the full extent of the comparison refer to 
Section~\ref{sec:compresults}.

Only two tables of the aforementioned resulting dataset tables have been 
included in this document, as an example of the results obtained through the
experiment.

The first case is the table obtained from Apache dataset (see 
Table~\ref{tab:kfold}).

\begin{center}
\begin{longtable}{ | r  l | c | c | c | c | c | c | c | }
\caption{K-Fold Metrics Tree Apache}\label{tab:kfold} \\

\hline
\textbf{\emph{Fold}} & \textbf{\emph{Function}} & \emph{Precision} & \emph{Recall}  & \emph{Fall Out} & \emph{Balanced} & \emph{F1} & \emph{MCC} & \emph{AUC} \\
\hline
\endfirsthead
\hline
\multicolumn{9}{c}{\tablename\ \thetable\ -- \textit{Continued from previous page}} \\
\hline
\textbf{\emph{Fold}} & \textbf{\emph{Function}} & \emph{Precision} & \emph{Recall}  & \emph{Fall Out} & \emph{Balanced} & \emph{F1} & \emph{MCC} & \emph{AUC} \\
\hline
\endhead
\hline
\multicolumn{9}{r}{\textit{Continued on next page}}
\endfoot
\hline
\endlastfoot

\multirow{3}{*}{\textbf{1}} & \textbf{Naive Bayes} & 
0.4286 & 0.1579 & 0.200  & 0.4789 & 0.2308 & -0.0548 & 0.4789 \\
& \textbf{Decision Tree} & 
0.8235 & 0.7368 & 0.1500 & 0.7934 & 0.7778 &  0.5915 & 0.7934 \\
& \textbf{Nearest Centroid} &
0.3333 & 0.3158 & 0.6000 & 0.3579 & 0.3243 & -0.2850 & 0.3579 \\
\hline
\multirow{3}{*}{\textbf{2}} & \textbf{Naive Bayes} & 
0.5000 & 0.0800 & 0.1538 & 0.4631 & 0.1379 & -0.1142 & 0.4631 \\
& \textbf{Decision Tree} & 
0.8095 & 0.6800 & 0.3077 & 0.6862 & 0.7391 &  0.3552 & 0.6862 \\
& \textbf{Nearest Centroid} &
0.5000 & 0.1600 & 0.3077 & 0.4262 & 0.2424 & -0.1719 & 0.4262 \\
\hline
\multirow{3}{*}{\textbf{3}} & \textbf{Naive Bayes} & 
0.5000 & 0.0909 & 0.1250 & 0.4830 & 0.1538 & -0.0548 & 0.4830 \\
& \textbf{Decision Tree} & 
0.8462 & 0.5000 & 0.1250 & 0.6875 & 0.6286 &  0.3903 & 0.6875  \\
& \textbf{Nearest Centroid} &
0.4000 & 0.1818 & 0.3750 & 0.4034 & 0.2500 & -0.2166 & 0.4034 \\
\hline
\multirow{3}{*}{\textbf{4}} & \textbf{Naive Bayes} & 
0.3333 & 0.8182 & 0.6667 & 0.5758 & 0.4737 & 0.1515 & 0.5758 \\
& \textbf{Decision Tree} & 
0.5833 & 0.6364 & 0.1852 & 0.7256 & 0.6087 & 0.4402 & 0.7256 \\
& \textbf{Nearest Centroid} &
0.8000 & 0.7273 & 0.0741 & 0.8266 & 0.7619 & 0.6727 & 0.8266 \\
\hline
\multirow{3}{*}{\textbf{5}} & \textbf{Naive Bayes} & 
0.3043 & 1.0000 & 0.5161 & 0.7419 & 0.4667 & 0.3838 & 0.7419 \\
& \textbf{Decision Tree} & 
0.5000 & 1.0000 & 0.2258 & 0.8871 & 0.6667 & 0.6222 & 0.8871 \\
& \textbf{Nearest Centroid} &
0.4545 & 0.7143 & 0.1935 & 0.7604 & 0.5556 & 0.4451 & 0.7604 \\
\hline
% Mean
\multirow{3}{*}{\textbf{Mean}} & \textbf{Naive Bayes} & 
0.4132 & 0.4294 & 0.3323 & 0.5485 & 0.2926 & 0.0623 & 0.5485 \\
& \textbf{Decision Tree} & 
0.7125 & 0.7106 & 0.1987 & 0.7560 & 0.6842 & 0.4799 & 0.7560 \\
& \textbf{Nearest Centroid} &
0.4976 & 0.4198 & 0.3101 & 0.5549 & 0.4268 & 0.0889 & 0.5549 \\
\hline
\end{longtable}
\end{center}

The second example is the table obtained on the experiment performed over the 
Hadoop (v0.8) dataset (see Table~\ref{tab:kfold2}).

\begin{center}
\begin{longtable}{ | r  l | c | c | c | c | c | c | c | }
\caption{K-Fold Metrics Tree Hadoop 0.8}\label{tab:kfold2} \\

\hline
\textbf{\emph{Fold}} & \textbf{\emph{Function}} & \emph{Precision} & \emph{Recall}  & \emph{Fall Out} & \emph{Balanced} & \emph{F1} & \emph{MCC} & \emph{AUC} \\
\hline
\endfirsthead
\hline
\multicolumn{9}{c}{\tablename\ \thetable\ -- \textit{Continued from previous page}} \\
\hline
\textbf{\emph{Fold}} & \textbf{\emph{Function}} & \emph{Precision} & \emph{Recall}  & \emph{Fall Out} & \emph{Balanced} & \emph{F1} & \emph{MCC} & \emph{AUC} \\
\hline
\endhead
\hline
\multicolumn{9}{r}{\textit{Continued on next page}}
\endfoot
\hline
\endlastfoot

\multirow{3}{*}{\textbf{1}} & \textbf{Naive Bayes} & 
1.0000 & 0.1250 & 0.0000 & 0.5625 & 0.2222 & 0.3262  & 0.5625 \\
& \textbf{Decision Tree} & 
0.0000 & 0.0000 & 0.0250 & 0.4875 & 0.0000 & -0.0652 & 0.4875 \\
& \textbf{Nearest Centroid} &
0.1667 & 1.0000 & 1.0000 & 0.5000 & 0.2857 & 0.0000  & 0.5000 \\
\hline
\multirow{3}{*}{\textbf{2}} & \textbf{Naive Bayes} & 
0.0000 & 0.0000 & 0.1064 & 0.4468 & 0.0000 & -0.0497 & 0.4468 \\
& \textbf{Decision Tree} & 
0.0000 & 0.0000 & 0.1064 & 0.4468 & 0.0000 & -0.0497 & 0.4468 \\
& \textbf{Nearest Centroid} &
0.0208 & 1.0000 & 1.0000 & 0.5000 & 0.0408 & 0.0000 & 0.5000 \\
\hline
\multirow{3}{*}{\textbf{3}} & \textbf{Naive Bayes} & 
0.0000 & 0.0000 & 0.0851 & 0.4574 & 0.0000 & -0.0440 & 0.4574 \\
& \textbf{Decision Tree} & 
0.0000 & 0.0000 & 0.0851 & 0.4574 & 0.0000 & -0.0440 & 0.4574 \\
& \textbf{Nearest Centroid} &
0.0208 & 1.0000 & 1.0000 & 0.5000 & 0.0408 &  0.0000 & 0.5000 \\
\hline
\multirow{3}{*}{\textbf{4}} & \textbf{Naive Bayes} & 
0.3333 & 0.2500 & 0.0455 & 0.6023 & 0.2857 &  0.2335 & 0.6023 \\
& \textbf{Decision Tree} & 
0.0000 & 0.0000 & 0.0426 & 0.4787 & 0.0000 & -0.0304 & 0.4787 \\
& \textbf{Nearest Centroid} &
0.0000 & 0.0000 & 0.0227 & 0.4886 & 0.0000 & -0.0440 & 0.4886 \\
\hline
\multirow{3}{*}{\textbf{5}} & \textbf{Naive Bayes} & 
0.0000 & 0.0000 & 0.0000 & 0.5000 & 0.0000 & 0.0000 & 0.5000 \\
& \textbf{Decision Tree} & 
0.0000 & 0.0000 & 0.0000 & 0.5000 & 0.0000 & 0.0000 & 0.5000 \\
& \textbf{Nearest Centroid} &
0.0000 & 0.0000 & 0.0000 & 0.5000 & 0.0000 & 0.0000 & 0.5000 \\
\hline
% Mean
\multirow{3}{*}{\textbf{Mean}} & \textbf{Naive Bayes} & 
0.26667 & 0.0750 & 0.0474 & 0.5138 & 0.1016 & 0.0932 & 0.5138 \\
& \textbf{Decision Tree} & 
0.0000 & 0.0000 & 0.0484 & 0.4758 & 0.0000 & -0.0446 & 0.4758 \\
& \textbf{Nearest Centroid} &
0.0417 & 0.6000 & 0.6045 & 0.4977 & 0.0735 & -0.0088 & 0.4977 \\
\hline
\end{longtable}
\end{center}

The inner results of each fold are not evaluated when analyzing the results
of the experiments, but that data could be used to understand how does
imbalance exactly affect classification. The objective evaluated in this 
paper is to see if the metrics obtained for each dataset have any relation with 
the performance of the classifiers. 

%%%%%%%%%%%%%%%%%%%%%%%%%%%%%%%%%%%%%%%%%%%%%%%%%%%%%%%%%%%%%%%%%%%%%%
\subsection{Filters affect Classification}

After experimenting with K-folding, it is time to see if filters applied to
the dataset have any effect on the quality of the trained classifier. In this 
case it is carried out by applying imbalance filters to the folds in the
datasets (see Section~\ref{sec:imbtechniques} for more information on 
imbalance filters).

The main target is, as the avid reader can guess, imbalanced datasets and how
filters that affect imbalance also affect the performance and other metrics of 
classifiers - results are going to be compared on Section~\ref{sec:compresults}.

Out of all the aforementioned techniques (\ref{sec:imbtechniques}) available 
for imbalanced datasets, only the two following techniques are going to be used 
in the experiments:

\begin{description}
    \item [Undersampling] Removing samples from majority class.
    \item [Oversampling] Creating new samples in the minority class.
\end{description}

These techniques have been selected as they are more than enough to see if
classifiers are affected by them. These two techniques have been used in two 
separate experiments.

%%%%%%%%%%%%%%%%%%%%%%%%%%%%%%%%%%%%%%%%%%%%%%%%%%%%%%%%%%%%
\subsubsection{Undersampling and $K$-fold on Classification}
\label{sec:kfoldunder}

For this experiment, one undersampling algorithm is applied 
(\ref{nearmiss}) to see how folds are affected by removing majority class
samples, and therefore, looking at how it impacts the classification algorithm 
being used is . 

The undersampling technique is not done on the whole dataset. Actually, the
folds are the ones affected by this algorithm. That way, instead of removing the
imbalance in the whole dataset, it is removed in each fold as necessary,
as some folds might have different imbalance level: what might be the majority 
class in the whole dataset can be the minority class of a certain fold.

The value selected for K-fold Cross Validation is $k=5$. A simplification
of the procedure followed by this experiment can be found in the Algorithm~\ref{alg:kfoldunder}.

\begin{breakablealgorithm}
    \caption{Undersampling and K-fold on classification performance}
    \label{alg:kfoldunder}
    \begin{algorithmic}[1]
        \Require $\mathcal{D}$ dataset
        \State inputs, targets $\leftarrow$ $\mathcal{D}$
    
        \State k $\leftarrow$ 5
        \State kf $\leftarrow$ kfold(k)
        \State splits $\leftarrow$ kf.split(inputs)
        
        \State samplingStrategy $\leftarrow$ {0:50, 1:50} \Comment{Binary class with a post undersampling distribution of 50-50}
        \State randomState $\leftarrow$ 42
        
        \ForAll {$train_{i}$, $test_{i}$ in splits}
        	\State $x_{train}$, $x_{test}$ $\leftarrow$ inputs.get($train_{i}$), inputs.get($test_{i}$)
        	\State $y_{train}$, $y_{test}$ $\leftarrow$ targets.get($train_{i}$), targets.get($test_{i}$)
        	
        	\State X, Y = makeImbalance(
        	\State \quad $x_{train}$, $y_{train}$,
        	\State \quad samplingStrategy,
        	\State \quad randomState
        	\State )
        	
        	\State clf $\leftarrow$ classifier(NearMiss(2), GaussianNB())
        	\State clf $\leftarrow$ trainNetwork(X, Y)
        	
        	\State confusionMatrix($x_{test}$, $y_{test}$, clf)
        \EndFor
        \State calculateMeanPerformance(clf)
    \end{algorithmic}
\end{breakablealgorithm}

Just as before, more than one classification algorithm has been used to measure
performance of the alterations done on the dataset - (1) Naive Bayes 
(Gaussian); (2) Decision Tree; and (3) kNN Nearest Centroid.

The experiment creates the same tables obtained in 
experiment~\ref{sec:exp-kfold}, that is why sharing all the resulting tables of 
this experiment would be unnecessary - the results evaluated in this paper
can be found in Section~\ref{sec:compresults}. That is why only two of them 
have been included into this document. The Apache dataset experiment (see 
Table~\ref{tab:kfoldunder}).

\begin{center}
\begin{longtable}{ | r  l | c | c | c | c | c | c | c | }
\caption{K-Fold Metrics with Undersampling Results Apache}
\label{tab:kfoldunder} 
\\

\hline
\textbf{\emph{Fold}} & \textbf{\emph{Function}} & 
\emph{Precision} & \emph{Recall}  & \emph{Fall Out} & 
\emph{Balanced} & \emph{F1} & \emph{MCC} & \emph{AUC} \\
\hline
\endfirsthead
\hline
\multicolumn{9}{c}{\tablename\ \thetable\ -- \textit{Continued from previous page}} \\
\hline
\textbf{\emph{Fold}} & \textbf{\emph{Function}} & 
\emph{Precision} & \emph{Recall}  & \emph{Fall Out} & 
\emph{Balanced} & \emph{F1} & \emph{MCC} & \emph{AUC} \\
\hline
\endhead
\hline
\multicolumn{9}{r}{\textit{Continued on next page}}
\endfoot
\hline
\endlastfoot
\hline

\multirow{3}{*}{\textbf{1}} & \textbf{Naive Bayes} & 
0.5556 & 0.8824 & 0.5455 & 0.6684 & 0.6818 &  0.3620 & 0.6684 \\
& \textbf{Decision Tree} & 
0.7000 & 0.8235 & 0.2727 & 0.7754 & 0.7568 &  0.5464 & 0.7754 \\
& \textbf{Nearest Centroid} &
0.4167 & 0.2941 & 0.3182 & 0.4880 & 0.3448 & -0.0259 & 0.4880 \\
\hline
\multirow{3}{*}{\textbf{2}} & \textbf{Naive Bayes} & 
0.5500 & 0.7333 & 0.3913 & 0.6710 & 0.6286 &  0.3348 & 0.6710 \\
& \textbf{Decision Tree} & 
0.5500 & 0.7333 & 0.3913 & 0.6710 & 0.6286 &  0.3348 & 0.6710 \\
& \textbf{Nearest Centroid} &
0.3571 & 0.3333 & 0.3913 & 0.4710 & 0.3448 & -0.0587 & 0.4710  \\
\hline
\multirow{3}{*}{\textbf{3}} & \textbf{Naive Bayes} & 
0.7586 & 0.8800 & 0.5385 & 0.6708 & 0.8148 & 0.3811 & 0.6708 \\
& \textbf{Decision Tree} & 
0.8421 & 0.6400 & 0.2308 & 0.7046 & 0.7273 & 0.3883 & 0.7046 \\
& \textbf{Nearest Centroid} &
0.7000 & 0.2800 & 0.2308 & 0.5246 & 0.4000 & 0.0530 & 0.5246 \\
\hline
\multirow{3}{*}{\textbf{4}} & \textbf{Naive Bayes} & 
0.6667 & 0.7273 & 0.1481 & 0.7896 & 0.6957 & 0.5650 & 0.7896 \\
& \textbf{Decision Tree} & 
0.6667 & 0.7273 & 0.1481 & 0.7896 & 0.6957 & 0.5650 & 0.7896 \\
& \textbf{Nearest Centroid} &
0.3077 & 0.3636 & 0.3333 & 0.5152 & 0.3333 & 0.0290 & 0.5152 \\
\hline
\multirow{3}{*}{\textbf{5}} & \textbf{Naive Bayes} & 
0.5000 & 0.5625 & 0.4091 & 0.5767 & 0.5294 & 0.1517 & 0.5767 \\
& \textbf{Decision Tree} & 
0.6316 & 0.7500 & 0.3182 & 0.7159 & 0.6857 & 0.4264 & 0.7159 \\
& \textbf{Nearest Centroid} &
0.5556 & 0.3125 & 0.1818 & 0.5653 & 0.4000 & 0.1518 & 0.5653 \\
\hline
\multirow{3}{*}{\textbf{Mean}} & \textbf{Naive Bayes} & 
0.5928 & 0.6662 & 0.3917 & 0.6372 & 0.6059 & 0.2992 & 0.6372 \\
& \textbf{Decision Tree} & 
0.6781 & 0.7348 & 0.2722 & 0.7313 & 0.6988 & 0.4522 & 0.7313 \\
& \textbf{Nearest Centroid} &
0.4674 & 0.3167 & 0.2911 & 0.5128 & 0.3646 & 0.0298 & 0.5128 \\
\hline
\end{longtable}
\end{center}

And the result obtained on the same experiment for the Hadoop (v0.8)
experiment (see Table~\ref{tab:kfoldunder2}).

\begin{center}
\begin{longtable}{ | r  l | c | c | c | c | c | c | c | }
\caption{K-Fold Metrics with Undersampling Results Hadoop 0.8}\label{tab:kfoldunder2} \\

\hline
\textbf{\emph{Fold}} & \textbf{\emph{Function}} & 
\emph{Precision} & \emph{Recall}  & \emph{Fall Out} & 
\emph{Balanced} & \emph{F1} & \emph{MCC} & \emph{AUC} \\
\hline
\endfirsthead
\hline
\multicolumn{9}{c}{\tablename\ \thetable\ -- \textit{Continued from previous page}} \\
\hline
\textbf{\emph{Fold}} & \textbf{\emph{Function}} & 
\emph{Precision} & \emph{Recall}  & \emph{Fall Out} & 
\emph{Balanced} & \emph{F1} & \emph{MCC} & \emph{AUC} \\
\hline
\endhead
\hline
\multicolumn{9}{r}{\textit{Continued on next page}}
\endfoot
\hline
\endlastfoot
\hline

\multirow{3}{*}{\textbf{1}} & \textbf{Naive Bayes} & 
0.1429 & 0.5000 & 0.1304 & 0.6848 & 0.2222 &  0.2092 & 0.6848 \\
& \textbf{Decision Tree} & 
0.1053 & 1.0000 & 0.3696 & 0.8152 & 0.1905 &  0.2576 & 0.8152 \\
& \textbf{Nearest Centroid} &
0.0345 & 0.5000 & 0.6087 & 0.4457 & 0.0645 & -0.0444 & 0.4457 \\
\hline
\multirow{3}{*}{\textbf{2}} & \textbf{Naive Bayes} & 
0.0000 & 0.0000 & 0.2667 & 0.3667 & 0.0000 & -0.1491 & 0.3667 \\
& \textbf{Decision Tree} & 
0.0833 & 0.3333 & 0.2444 & 0.5444 & 0.1333 &  0.0497 & 0.5444 \\
& \textbf{Nearest Centroid} &
0.0952 & 0.6667 & 0.4222 & 0.6222 & 0.1667 &  0.1193 & 0.6222 \\
\hline
\multirow{3}{*}{\textbf{3}} & \textbf{Naive Bayes} & 
0.0833 & 0.5000 & 0.2391 & 0.6304 & 0.1429 &  0.1204 & 0.6304\\
& \textbf{Decision Tree} & 
0.0000 & 0.0000 & 0.3261 & 0.3370 & 0.0000 & -0.1406 & 0.3370 \\
& \textbf{Nearest Centroid} &
0.0000 & 0.0000 & 0.4348 & 0.2826 & 0.0000 & -0.1762 & 0.2826 \\
\hline
\multirow{3}{*}{\textbf{4}} & \textbf{Naive Bayes} & 
0.1333 & 0.3333 & 0.3095 & 0.5119 & 0.1905 &  0.0170 & 0.5119 \\
& \textbf{Decision Tree} & 
0.1765 & 0.5000 & 0.3333 & 0.5833 & 0.2609 &  0.1153 & 0.5833 \\
& \textbf{Nearest Centroid} &
0.0625 & 0.1667 & 0.3571 & 0.4048 & 0.0909 & -0.1336 & 0.4048 \\
\hline
\multirow{3}{*}{\textbf{5}} & \textbf{Naive Bayes} & 
0.0000 & 0.0000 & 0.1556 & 0.4222 & 0.0000 & -0.1067 & 0.4222 \\
& \textbf{Decision Tree} & 
0.2000 & 0.6667 & 0.1778 & 0.7444 & 0.3077 &  0.2914 & 0.7444 \\
& \textbf{Nearest Centroid} &
0.0000 & 0.0000 & 0.2444 & 0.3778 & 0.0000 & -0.1408 & 0.3778 \\
\hline
\multirow{3}{*}{\textbf{Mean}} & \textbf{Naive Bayes} & 
0.0719 & 0.2667 & 0.2203 & 0.5232 & 0.1111 &  0.0182 & 0.5232 \\
& \textbf{Decision Tree} & 
0.1130 & 0.5000 & 0.2902 & 0.6049 & 0.1785 &  0.1147 & 0.6049 \\
& \textbf{Nearest Centroid} &
0.0384 & 0.2667 & 0.4134 & 0.4266 & 0.0644 & -0.0751 & 0.4266 \\
\hline
\end{longtable}
\end{center}

As it has been mentioned in Section~\ref{nearmiss}, there are several versions
of \textit{Near Miss} algorithm. For this experiment it has been used 
\textit{version 2}. This undersampling algorithm selects the majority class
samples with the least average distance to the 3 farthest minority class 
samples.

The analysis of the metrics obtained are in Section~\ref{sec:compresults}.

%%%%%%%%%%%%%%%%%%%%%%%%%%%%%%%%%%%%%%%%%%%%%%%%%%%%%%%%%%%%
\subsubsection{Oversampling and K-fold on Classification}

In this experiment, similarly to Section~\ref{sec:kfoldunder}, an imbalance 
filter is included to the K-fold Cross Validation process. In this case, the 
oversampling technique is implemented and is later on compared to other 
experiment results (see Section~\ref{sec:compresults}).

The technique has been applied to the different folds of the dataset, not to 
the entire data. Once again, the dataset is divided into $k=5$ folds. 

The logic of the experiment is summarized in the next script (see 
Algorithm~\ref{alg:kfoldover}). The difference with the
undersampling Algorithm~\ref{alg:kfoldunder} is that uses SMOTE technique (see 
Section~\ref{smote}) instead of \textit{Near Miss} undersampling technique.

\begin{breakablealgorithm}
    \caption{Oversampling and K-fold on classification performance}
    \footnotesize
    \label{alg:kfoldover}
    \begin{algorithmic}[1]
        \Require $\mathcal{D}$ dataset
        \State inputs, targets $\leftarrow$ $\mathcal{D}$
    
        \State k $\leftarrow$ 5
        \State kf $\leftarrow$ kfold(k)
        \State splits $\leftarrow$ kf.split(inputs)
        
        \ForAll {$train_{i}$, $test_{i}$ in splits}
        	\State $x_{train}$, $x_{test}$ $\leftarrow$ inputs.get($train_{i}$), inputs.get($test_{i}$)
        	\State $y_{train}$, $y_{test}$ $\leftarrow$ targets.get($train_{i}$), targets.get($test_{i}$)
        	
        	\State clf $\leftarrow$ classifier(SMOTE(), GaussianNB())
        	\State clf $\leftarrow$ trainNetwork($x_{train}$, $y_{train}$)
        	
        	\State confusionMatrix($x_{test}$, $y_{test}$, clf)
        \EndFor
        \State calculateMeanPerformance(clf)
    \end{algorithmic}
\end{breakablealgorithm}

The classifiers used for this experiment are - (1) Naive Bayes (Gaussian); 
(2) Decision Tree; and (3) kNN Nearest Centroid.

Not all the resulting tables can be shown, so only two examples have been 
included. The first one is the result on the Apache dataset (see 
Table~\ref{tab:kfoldover-apache}).

\begin{center}
\begin{longtable}{ | r  l | c | c | c | c | c | c | c | }
\caption{K-Fold Metrics with Oversampling Results Apache}
\label{tab:kfoldover-apache} 
\\

\hline
\textbf{\emph{Fold}} & \textbf{\emph{Function}} & 
\emph{Precision} & \emph{Recall}  & \emph{Fall Out} & 
\emph{Balanced} & \emph{F1} & \emph{MCC} & \emph{AUC} \\
\hline
\endfirsthead
\hline
\multicolumn{9}{c}{\tablename\ \thetable\ -- \textit{Continued from previous page}} \\
\hline
\textbf{\emph{Fold}} & \textbf{\emph{Function}} & 
\emph{Precision} & \emph{Recall}  & \emph{Fall Out} & 
\emph{Balanced} & \emph{F1} & \emph{MCC} & \emph{AUC} \\
\hline
\endhead
\hline
\multicolumn{9}{r}{\textit{Continued on next page}}
\endfoot
\hline
\endlastfoot

\multirow{3}{*}{\textbf{1}} & \textbf{Naive Bayes} & 
0.4286 & 0.1579 & 0.2000 & 0.4789 & 0.2308 & -0.0548 & 0.4789 \\
& \textbf{Decision Tree} & 
0.7222 & 0.6842 & 0.2500 & 0.7171 & 0.7027 &  0.4354 & 0.7171 \\
& \textbf{Nearest Centroid} &
0.3333 & 0.3158 & 0.6000 & 0.3579 & 0.3243 & -0.2850 & 0.3579 \\
\hline
\multirow{3}{*}{\textbf{2}} & \textbf{Naive Bayes} & 
0.5000 & 0.0800 & 0.1538 & 0.4631 & 0.1379 & -0.1142 & 0.4631 \\
& \textbf{Decision Tree} & 
0.8889 & 0.6400 & 0.1538 & 0.7431 & 0.7442 &  0.4619 & 0.7431 \\
& \textbf{Nearest Centroid} &
0.5000 & 0.1600 & 0.3077 & 0.4262 & 0.2424 & -0.1719 & 0.4262 \\
\hline
\multirow{3}{*}{\textbf{3}} & \textbf{Naive Bayes} & 
0.5000 & 0.0909 & 0.1250 & 0.4830 & 0.1538 & -0.0548 & 0.4830 \\
& \textbf{Decision Tree} & 
0.8000 & 0.5455 & 0.1875 & 0.6790 & 0.6486 &  0.3616 & 0.6790 \\
& \textbf{Nearest Centroid} &
0.4286 & 0.1364 & 0.2500 & 0.4432 & 0.2069 & -0.1447 & 0.4432 \\
\hline
\multirow{3}{*}{\textbf{4}} & \textbf{Naive Bayes} & 
0.3333 & 0.8182 & 0.6667 & 0.5758 & 0.4737 &  0.1515 & 0.5758 \\
& \textbf{Decision Tree} & 
0.5385 & 0.6364 & 0.2222 & 0.7071 & 0.5833 &  0.3959 & 0.7071 \\
& \textbf{Nearest Centroid} &
0.6000 & 0.5455 & 0.1481 & 0.6987 & 0.5714 &  0.4092 & 0.6987 \\
\hline
\multirow{3}{*}{\textbf{5}} & \textbf{Naive Bayes} & 
0.3182 & 1.0000 & 0.4839 & 0.7581 & 0.4828 & 0.4052 & 0.7581 \\
& \textbf{Decision Tree} & 
0.5000 & 1.0000 & 0.2258 & 0.8871 & 0.6667 & 0.6222 & 0.8871 \\
& \textbf{Nearest Centroid} &
0.4545 & 0.7143 & 0.1935 & 0.7604 & 0.5556 & 0.4451 & 0.7604 \\
\hline
% Mean
\multirow{3}{*}{\textbf{Mean}} & \textbf{Naive Bayes} & 
0.4160 & 0.4294 & 0.3259 & 0.5518 & 0.2958 & 0.0666 & 0.5518 \\
& \textbf{Decision Tree} & 
0.6899 & 0.7012 & 0.2079 & 0.7467 & 0.6691 & 0.4554 & 0.7467 \\
& \textbf{Nearest Centroid} &
0.4633 & 0.3744 & 0.2999 & 0.5373 & 0.3801 & 0.0505 & 0.5373 \\
\hline

\end{longtable}
\end{center}

The second example is the one from the Hadoop (v0.8) experiment (see 
Table~\ref{tab:kfoldover-hadoop}).

\begin{center}
\begin{longtable}{ | r  l | c | c | c | c | c | c | c | }
\caption{K-Fold Metrics with Oversampling Results Hadoop 0.8}\label{tab:kfoldover-hadoop} \\

\hline
\textbf{\emph{Fold}} & \textbf{\emph{Function}} & \emph{Precision} & \emph{Recall}  & \emph{Fall Out} & \emph{Balanced} & \emph{F1} & \emph{MCC} & \emph{AUC} \\
\hline
\endfirsthead
\hline
\multicolumn{9}{c}{\tablename\ \thetable\ -- \textit{Continued from previous page}} \\
\hline
\textbf{\emph{Fold}} & \textbf{\emph{Function}} & \emph{Precision} & \emph{Recall}  & \emph{Fall Out} & \emph{Balanced} & \emph{F1} & \emph{MCC} & \emph{AUC} \\
\hline
\endhead
\hline
\multicolumn{9}{r}{\textit{Continued on next page}}
\endfoot
\hline
\endlastfoot

\multirow{3}{*}{\textbf{1}} & \textbf{Naive Bayes} & 
0.1176 & 0.2500 & 0.3750 & 0.4375 & 0.1600 & -0.0974 & 0.4375 \\
& \textbf{Decision Tree} & 
0.2500 & 0.1250 & 0.0750 & 0.5250 & 0.1667 &  0.0674 & 0.5250 \\
& \textbf{Nearest Centroid} &
0.1667 & 1.0000 & 1.0000 & 0.5000 & 0.2857 &  0.0000 & 0.5000 \\
\hline
\multirow{3}{*}{\textbf{2}} & \textbf{Naive Bayes} & 
0.1000 & 1.0000 & 0.1915 & 0.9043 & 0.1818 &  0.2843 & 0.9043 \\
& \textbf{Decision Tree} & 
0.0000 & 0.0000 & 0.4468 & 0.2766 & 0.0000 & -0.1286 & 0.2766 \\
& \textbf{Nearest Centroid} &
0.0208 & 1.0000 & 1.0000 & 0.5000 & 0.0408 &  0.0000 & 0.5000 \\
\hline
\multirow{3}{*}{\textbf{3}} & \textbf{Naive Bayes} & 
0.0000 & 0.0000 & 0.1915 & 0.4043 & 0.0000 & -0.0701 & 0.4043 \\
& \textbf{Decision Tree} & 
0.0000 & 0.0000 & 0.1064 & 0.4468 & 0.0000 & -0.0497 & 0.4468 \\
& \textbf{Nearest Centroid} &
0.0208 & 1.0000 & 1.0000 & 0.5000 & 0.0408 &  0.0000 & 0.5000 \\
\hline
\multirow{3}{*}{\textbf{4}} & \textbf{Naive Bayes} & 
0.0667 & 0.5000 & 0.6364 & 0.4318 & 0.1176 & -0.0778 & 0.4318 \\
& \textbf{Decision Tree} & 
0.2500 & 0.2500 & 0.0682 & 0.5909 & 0.2500 &  0.1818 & 0.5909 \\
& \textbf{Nearest Centroid} &
0.0000 & 0.0000 & 0.0000 & 0.5000 & 0.0000 &  0.0000 & 0.5000 \\
\hline
\multirow{3}{*}{\textbf{5}} & \textbf{Naive Bayes} & 
0.0000 & 0.0000 & 0.0000 & 0.5000 & 0.0000 &  0.0000 & 0.5000 \\
& \textbf{Decision Tree} & 
0.0000 & 0.0000 & 0.0435 & 0.4783 & 0.0000 & -0.0435 & 0.4783 \\
& \textbf{Nearest Centroid} &
0.0000 & 0.0000 & 0.0000 & 0.5000 & 0.0000 &  0.0000 & 0.5000 \\
\hline
\multirow{3}{*}{\textbf{Mean}} & \textbf{Naive Bayes} & 
0.0569 & 0.3500 & 0.2789 & 0.5356 & 0.0919 &  0.0078 & 0.5356 \\
& \textbf{Decision Tree} & 
0.1000 & 0.0750 & 0.1480 & 0.4635 & 0.0833 &  0.0055 & 0.4635 \\
& \textbf{Nearest Centroid} &
0.0417 & 0.6000 & 0.6000 & 0.5000 & 0.0735 &  0.0000 & 0.5000 \\
\hline
\end{longtable}
\end{center}

The analysis of the metrics obtained are in Section~\ref{sec:compresults}.

%%%%%%%%%%%%%%%%%%%%%%%%%%%%%%%%%%%%%%%%%%%%%%%%%%%%%%%%%%%%%%%%%%%%%%
\subsection{Compare Results}\label{sec:compresults}

In this part of the paper, we are going to analyze the results coming from the
previous experiments. These are displayed in tables \ref{tab:all-metrics}, 
\ref{tab:all-metrics2} and \ref{tab:all-metrics3}. 

To answer RQ1 and RQ2 (\ref{q:rq1} and \ref{q:rq2}, respectively) we need to 
observe, for example, the values from \textit{overlapping}, we assume that 
there is a certain relation with the classification performance. For most cases 
in our experimentation, the higher the overlapping rate, the less efficient the 
classifier becomes. Of course, this assumption is not backed up by any 
statistical experiment on the datasets and/or the classifiers, but it looking 
at the results it looks right at first sight.

For example, looking at the \textit{overlapping} measures, when 
\textit{overlapping F1} is higher than 0.90, almost none of the classifiers is 
able to surpass 0.5 in \textit{precision} score, something similar happens with 
the \textit{recall} an \textit{fall-out} scores (all three measures indicate on 
a relative value how many input samples are correctly predicted). It makes 
sense once the reader realizes that the more interlaced the samples are (the 
more \textit{overlapping}), it should be harder for a classifier to identify 
what data belongs to each class (RQ2 - see \ref{q:rq2}).

In a similar way, it \textit{linearity} measures (RQ1 - see \ref{q:rq1}) are 
directly related to the performance of the classification algorithms. Having 
high \textit{overlapping} measures means that the \textit{linearity} should be 
smaller (it is harder to separate data using a linear function if data is 
interlaced) - RQ2 (see \ref{q:rq2}). Therefore, the higher the 
\textit{linearity} measures obtained, the higher the performance should be. No 
examples of this behavior can be shown, as all the selected datasets seem to 
have high \textit{overlapping} values.

A third complexity metric that should be taken into account, regarding 
classifiers performance, is \textit{balance} - (RQ1 and RQ3, \ref{q:rq1} and
\ref{q:rq3}, respectively). Its measures try to approximate the level of 
imbalance of a certain class. Therefore, the higher the measures' values, the 
greater should be the gap between the number of samples from the majority and 
minority class (all the datasets work with binary classes).

It can be observed that the greater the imbalances, not only the results have 
poor performance (less than 0.7 of \textit{precision}, and similarly with 
\textit{recall} and \textit{fall-out} measures), but it also means that the 
different classifiers have very different results on the same dataset. This can 
be translated as, the performance on certain classifiers might depend on the 
imbalance of a certain dataset (RQ2 - see \ref{q:rq2}).

Taking a look at \textit{Balanced Accuracy} measure (useful for imbalanced 
datasets, like the ones used for the experiments on this paper), it can be 
observed that most of the classifiers do not behave as they should in either
the positive or negative prediction. In example, all of the hadoop datasets,
where the imbalance ratio (\textit{balance} measures) is not ideal, and the
overlapping is too high. 

Therefore, \textit{MCC}, as a measure to see the quality of a binary classifier
should also be low - just like in hadoop datasets.

As for this last observation, it could be assumed that filters that would reduce
imbalance, should also be able to increase the performance of the classifiers.

After some experimentation regarding this matter, no clear improvements have
been noticed in any of the analytical measures. Furthermore, undersampling
seems not to be a good technique when the number of samples is too low, as the
number of samples in the resulting training dataset is not enough to obtain a
good classifier. In some cases, it even reduces the performance.

After using SMOTE oversampling technique, the \textit{Balanced Accuracy} metric
seems to indicate that the classifier identified slightly better some of the 
testing data samples, but the resulting classifier still does not meet the
desired quality (\textit{MCC}).

But, overall, the techniques applied do not seem to fix perfectly the problems
arose from either imbalance or overlapped datasets. This topic is going to be
further discussed in Section~\ref{sec:conclusions}, as well as some other 
remarks about the selected classifiers and possible further experimentation on
this topic.

On trying to answer the final RQ, \textit{Do complexity metrics tell us 
something about the quality of the dataset?} (see \ref{q:rq4}), we can assume 
(looking at the results), they do. A dataset with bad quality could be 
considered one that has high overlapping, imbalanced classes, etc. As the 
results seem to indicate a relation between those values and the performance of
the obtained classifier - as the analytical metrics tell us.

\lhead{}

\begin{landscape}
\begin{center}
\begin{tiny}
\begin{longtable}{ | r  l | c | c | c | 
                            c | c | c | 
                            c | c | c | 
                            c | c | c | 
                            c | c | c | 
                            c | c | c | }
\caption{All Metrics}\label{tab:all-metrics} \\

\hline
\textbf{\textit{Measure}}                        & 
\textbf{\textit{Metr.}}                          & 
\multicolumn{3}{| c |}{\textbf{\textit{ant}}}    & 
\multicolumn{3}{| c |}{\textbf{\textit{apache}}} & 
\multicolumn{3}{| c |}{\textbf{\textit{camel}}}  &
\multicolumn{3}{| c |}{\textbf{\textit{ivy}}}    &
\multicolumn{3}{| c |}{\textbf{\textit{jedit}}}  &
\multicolumn{3}{| c |}{\textbf{\textit{log4j}}} \\

\hline
\endfirsthead
\hline
\multicolumn{14}{c}{\tablename\ \thetable\ -- \textit{Continued from previous page}} \\
\hline

\\
\hline
\endhead
\hline
\textbf{\textit{Measure}}                        & 
\textbf{\textit{Metr.}}                          & 
\multicolumn{3}{| c |}{\textbf{\textit{ant}}}    & 
\multicolumn{3}{| c |}{\textbf{\textit{apache}}} & 
\multicolumn{3}{| c |}{\textbf{\textit{camel}}}  &
\multicolumn{3}{| c |}{\textbf{\textit{ivy}}}    &
\multicolumn{3}{| c |}{\textbf{\textit{jedit}}}  &
\multicolumn{3}{| c |}{\textbf{\textit{log4j}}} \\
\endfoot
\hline
\endlastfoot

\multirow{2}{*}{\emph{\textbf{Balance}}} & \emph{C1} & 
\multicolumn{3}{| c |}{0.7653} & 
\multicolumn{3}{| c |}{0.9895} & 
\multicolumn{3}{| c |}{0.7114} & 
\multicolumn{3}{| c |}{0.5108} & 
\multicolumn{3}{| c |}{0.1545} & 
\multicolumn{3}{| c |}{0.3953} 
\\

& \emph{C2} & 
\multicolumn{3}{| c |}{0.4701} & 
\multicolumn{3}{| c |}{0.0286} & 
\multicolumn{3}{| c |}{0.5429} & 
\multicolumn{3}{| c |}{0.7477} & 
\multicolumn{3}{| c |}{0.9543} & 
\multicolumn{3}{| c |}{0.8319}
\\ 

\hline
\multirow{3}{*}{\emph{\textbf{Correlation}}} & \emph{C2} & 
\multicolumn{3}{| c |}{0.2622} & 
\multicolumn{3}{| c |}{0.0891} &
\multicolumn{3}{| c |}{0.1227} &
\multicolumn{3}{| c |}{0.1879} & 
\multicolumn{3}{| c |}{0.0448} & 
\multicolumn{3}{| c |}{0.0804}
\\

& \emph{C3} & 
\multicolumn{3}{| c |}{0.6338} & 
\multicolumn{3}{| c |}{0.7372} & 
\multicolumn{3}{| c |}{0.5763} & 
\multicolumn{3}{| c |}{0.5588} & 
\multicolumn{3}{| c |}{0.5324} & 
\multicolumn{3}{| c |}{0.5780}
\\

& \emph{C4} & 
\multicolumn{3}{| c |}{0.7409} &
\multicolumn{3}{| c |}{1.0000} & 
\multicolumn{3}{| c |}{1.0000} &
\multicolumn{3}{| c |}{0.3722} &
\multicolumn{3}{| c |}{0.0285} &
\multicolumn{3}{| c |}{0.4341}
\\

\hline
\multirow{2}{*}{\emph{\textbf{Dimension.}}} & \emph{T2} & 
\multicolumn{3}{| c |}{0.0268} & 
\multicolumn{3}{| c |}{0.0524} & 
\multicolumn{3}{| c |}{0.0207} &
\multicolumn{3}{| c |}{0.0568} &
\multicolumn{3}{| c |}{0.0407} &
\multicolumn{3}{| c |}{0.0976}
\\

& \emph{T3} & 
\multicolumn{3}{| c |}{0.0027} &
\multicolumn{3}{| c |}{0.0105} &
\multicolumn{3}{| c |}{0.3000} &
\multicolumn{3}{| c |}{0.0057} &
\multicolumn{3}{| c |}{0.0020} &
\multicolumn{3}{| c |}{0.0098}
\\

\hline
\multirow{3}{*}{\emph{\textbf{Linearity}}} & \emph{L1} & 
\multicolumn{3}{| c |}{0.2477} &
\multicolumn{3}{| c |}{0.3937} &
\multicolumn{3}{| c |}{0.2774} & 
\multicolumn{3}{| c |}{0.1570} &
\multicolumn{3}{| c |}{0.0475} & 
\multicolumn{3}{| c |}{0.1426}
\\

& \emph{L2} & 
\multicolumn{3}{| c |}{0.1212} & 
\multicolumn{3}{| c |}{0.1917} & 
\multicolumn{3}{| c |}{0.1370} & 
\multicolumn{3}{| c |}{0.0726} & 
\multicolumn{3}{| c |}{0.0200} &
\multicolumn{3}{| c |}{0.0624}
\\

& \emph{L3} &
\multicolumn{3}{| c |}{0.1103} & 
\multicolumn{3}{| c |}{0.1828} & 
\multicolumn{3}{| c |}{0.1304} & 
\multicolumn{3}{| c |}{0.0661} & 
\multicolumn{3}{| c |}{0.0180} & 
\multicolumn{3}{| c |}{0.0593}
\\

\hline
\multirow{6}{*}{\emph{\textbf{Neighbor.}}} & \emph{N1} & 
\multicolumn{3}{| c |}{0.3208} &
\multicolumn{3}{| c |}{0.3822} & 
\multicolumn{3}{| c |}{0.3565} & 
\multicolumn{3}{| c |}{0.2159} & 
\multicolumn{3}{| c |}{0.0447} & 
\multicolumn{3}{| c |}{0.2195}
\\

& \emph{N2} & 
\multicolumn{3}{| c |}{0.3588} &
\multicolumn{3}{| c |}{0.3232} &
\multicolumn{3}{| c |}{0.3629} &
\multicolumn{3}{| c |}{0.3285} & 
\multicolumn{3}{| c |}{0.2543} &
\multicolumn{3}{| c |}{0.3588} 
\\

& \emph{N3} & 
\multicolumn{3}{| c |}{0.2067} & 
\multicolumn{3}{| c |}{0.1937} & 
\multicolumn{3}{| c |}{0.2435} & 
\multicolumn{3}{| c |}{0.1364} & 
\multicolumn{3}{| c |}{0.0325} &
\multicolumn{3}{| c |}{0.1220}
\\

& \emph{N4} & 
\multicolumn{3}{| c |}{0.0913} & 
\multicolumn{3}{| c |}{0.1728} & 
\multicolumn{3}{| c |}{0.1358} & 
\multicolumn{3}{| c |}{0.0739} & 
\multicolumn{3}{| c |}{0.0122} & 
\multicolumn{3}{| c |}{0.0293}
\\

& \emph{T1} & 
\multicolumn{3}{| c |}{0.0025} & 
\multicolumn{3}{| c |}{0.0135} & 
\multicolumn{3}{| c |}{0.0021} &
\multicolumn{3}{| c |}{0.0082} & 
\multicolumn{3}{| c |}{0.0184} & 
\multicolumn{3}{| c |}{0.0161}
\\

& \emph{LSC} & 
\multicolumn{3}{| c |}{0.9860} & 
\multicolumn{3}{| c |}{0.9659} &
\multicolumn{3}{| c |}{0.9923} & 
\multicolumn{3}{| c |}{0.9533} & 
\multicolumn{3}{| c |}{0.8870} & 
\multicolumn{3}{| c |}{0.9476}
\\

\hline
\multirow{3}{*}{\emph{\textbf{Network}}} & \emph{Density} & 
\multicolumn{3}{| c |}{0.8658} &
\multicolumn{3}{| c |}{0.8801} & 
\multicolumn{3}{| c |}{0.8627} & 
\multicolumn{3}{| c |}{0.8293} &
\multicolumn{3}{| c |}{0.8170} & 
\multicolumn{3}{| c |}{0.8260}
\\

& \emph{ClsCoef} & 
\multicolumn{3}{| c |}{0.3377} & 
\multicolumn{3}{| c |}{0.2232} & 
\multicolumn{3}{| c |}{0.3131} & 
\multicolumn{3}{| c |}{0.3564} & 
\multicolumn{3}{| c |}{0.3314} & 
\multicolumn{3}{| c |}{0.4158}
\\

& \emph{Hubs} & 
\multicolumn{3}{| c |}{0.7665} & 
\multicolumn{3}{| c |}{0.9236} & 
\multicolumn{3}{| c |}{0.7608} & 
\multicolumn{3}{| c |}{0.7628} & 
\multicolumn{3}{| c |}{0.8789} & 
\multicolumn{3}{| c |}{0.8487}
\\

\hline
\multirow{5}{*}{\emph{\textbf{Overlap}}} & \emph{F1} & 
\multicolumn{3}{| c |}{0.9192} & 
\multicolumn{3}{| c |}{0.9776} & 
\multicolumn{3}{| c |}{0.9825} & 
\multicolumn{3}{| c |}{0.9432} & 
\multicolumn{3}{| c |}{0.9874} & 
\multicolumn{3}{| c |}{0.9922}
\\

& \emph{F1v} & 
\multicolumn{3}{| c |}{0.2882} &
\multicolumn{3}{| c |}{0.4659} &
\multicolumn{3}{| c |}{0.5203} & 
\multicolumn{3}{| c |}{0.2083} & 
\multicolumn{3}{| c |}{0.2048} & 
\multicolumn{3}{| c |}{0.3223}
\\

& \emph{F2} & 
\multicolumn{3}{| c |}{0.0000} &
\multicolumn{3}{| c |}{0.0061} &
\multicolumn{3}{| c |}{0.0004} &
\multicolumn{3}{| c |}{0.0005} &
\multicolumn{3}{| c |}{0.0000} &
\multicolumn{3}{| c |}{0.0000}
\\

& \emph{F3} & 
\multicolumn{3}{| c |}{0.9074} & 
\multicolumn{3}{| c |}{0.9476} & 
\multicolumn{3}{| c |}{0.9513} & 
\multicolumn{3}{| c |}{0.7102} & 
\multicolumn{3}{| c |}{0.5915} & 
\multicolumn{3}{| c |}{0.6585}
\\

& \emph{F4} & 
\multicolumn{3}{| c |}{0.8295} & 
\multicolumn{3}{| c |}{0.8325} &
\multicolumn{3}{| c |}{0.8487} & 
\multicolumn{3}{| c |}{0.3977} & 
\multicolumn{3}{| c |}{0.0671} & 
\multicolumn{3}{| c |}{0.1659}
\\

\hline
\multirow{4}{*}{\emph{\textbf{Smooth.}}} & S1 & 
\multicolumn{3}{| c |}{0.2191} &
\multicolumn{3}{| c |}{0.2474} &
\multicolumn{3}{| c |}{0.2541} &
\multicolumn{3}{| c |}{0.1453} &
\multicolumn{3}{| c |}{0.0285} &
\multicolumn{3}{| c |}{0.1520}
\\

& \emph{S2} & 
\multicolumn{3}{| c |}{0.1410} &
\multicolumn{3}{| c |}{0.1178} &
\multicolumn{3}{| c |}{0.1389} &
\multicolumn{3}{| c |}{0.1460} &
\multicolumn{3}{| c |}{0.1417} & 
\multicolumn{3}{| c |}{0.1556}
\\

& \emph{S3} &
\multicolumn{3}{| c |}{0.2067} & 
\multicolumn{3}{| c |}{0.1937} &
\multicolumn{3}{| c |}{0.2466} & 
\multicolumn{3}{| c |}{0.1364} &
\multicolumn{3}{| c |}{0.0325} & 
\multicolumn{3}{| c |}{0.1220}
\\

& \emph{S4} & 
\multicolumn{3}{| c |}{0.0847} &
\multicolumn{3}{| c |}{0.1273} & 
\multicolumn{3}{| c |}{0.1608} & 
\multicolumn{3}{| c |}{0.0716} & 
\multicolumn{3}{| c |}{0.0215} & 
\multicolumn{3}{| c |}{0.0285}
\\

\hline
& &
\textit{NB} & \textit{DT} & \textit{NC} &
\textit{NB} & \textit{DT} & \textit{NC} &
\textit{NB} & \textit{DT} & \textit{NC} &
\textit{NB} & \textit{DT} & \textit{NC} &
\textit{NB} & \textit{DT} & \textit{NC} &
\textit{NB} & \textit{DT} & \textit{NC} 
\\

\hline
\multirow{3}{*}{\emph{\textbf{Precision}}} & 1 & 
0.5600 & 0.4598 & 0.6263 &
0.4132 & 0.7125 & 0.4976 & 
0.4478 & 0.3275 & 0.3304 &
0.2687 & 0.2533 & 0.3401 &
0.0645 & 0.1000 & 0.0496 &
0.9734 & 0.9350 & 0.9093 
\\

& \emph{2} & 
0.5093 & 0.3699 & 0.6367 &
0.5928 & 0.6781 & 0.4674 &
0.3937 & 0.2548 & 0.3099 &
0.2901 & 0.1836 & 0.3451 &
0.0179 & 0.0402 & 0.1127 &
0.9665 & 0.9630 & 0.9173 
\\

& \emph{3} & 
0.5152 & 0.4200 & 0.6166 &
0.4160 & 0.6899 & 0.4633 & 
0.4372 & 0.2891 & 0.3243 & 
0.2422 & 0.2478 & 0.3220 & 
0.0427 & 0.1686 & 0.0491 & 
0.9734 & 0.9510 & 0.9149 
\\

\hline
\multirow{3}{*}{\emph{\textbf{Recall}}} & 1 & 
0.5497 & 0.4677 & 0.5105 &
0.4294 & 0.7106 & 0.4198 &
0.2600 & 0.3732 & 0.2550 &
0.3122 & 0.2389 & 0.4372 &
0.1300 & 0.0500 & 0.1900 &
0.4912 & 0.9151 & 0.3216 
\\

& \emph{2} & 
0.6068 & 0.6416 & 0.4788 &
0.6662 & 0.7348 & 0.3167 &
0.2990 & 0.6038 & 0.2850 &
0.5508 & 0.6765 & 0.5000 &
0.3667 & 0.8500 & 0.4167 &
0.4786 & 0.5498 & 0.2804 
\\

& \emph{3} & 
0.5913 & 0.5115 & 0.5164 & 
0.4294 & 0.7012 & 0.3744 & 
0.3212 & 0.3528 & 0.3203 & 
0.4011 & 0.3122 & 0.4372 & 
0.1800 & 0.3800 & 0.1900 & 
0.5017 & 0.9261 & 0.3430 
\\

\hline
\multirow{3}{*}{\emph{\textbf{Fall Out}}} & 1 & 
0.1231 & 0.1552 & 0.0866 &
0.3323 & 0.1987 & 0.3101 &
0.0763 & 0.1838 & 0.1266 &
0.1104 & 0.0734 & 0.1008 &
0.1027 & 0.0226 & 0.1136 &
0.2167 & 0.7333 & 0.4000 
\\

& \emph{2} & 
0.1703 & 0.3126 & 0.0790 &
0.3917 & 0.2722 & 0.2911 &
0.1171 & 0.4313 & 0.1399 &
0.1603 & 0.3546 & 0.0996 &
0.3132 & 0.4359 & 0.0624 &
0.3200 & 0.2133 & 0.3533 
\\

& \emph{3} & 
0.1609 & 0.2051 & 0.0919 & 
0.3259 & 0.2079 & 0.2999 & 
0.0995 & 0.2072 & 0.1600 & 
0.1520 & 0.1248 & 0.1101 & 
0.2876 & 0.0643 & 0.1300 & 
0.2167 & 0.5333 & 0.4000 
\\

\hline
\multirow{3}{*}{\emph{\textbf{Balanced}}} & 1 & 
0.7133 & 0.6563 & 0.7120 &
0.5485 & 0.7560 & 0.5549 & 
0.5919 & 0.5947 & 0.5642 &
0.6009 & 0.5827 & 0.6682 &
0.6046 & 0.6097 & 0.6341 &
0.6373 & 0.5909 & 0.4608 
\\

& \emph{2} & 
0.7182 & 0.6645 & 0.6998 &
0.6372 & 0.7313 & 0.5128 &
0.5909 & 0.5862 & 0.5726 &
0.6952 & 0.6609 & 0.7002 &
0.5267 & 0.7071 & 0.6771 &
0.5793 & 0.6682 & 0.4635 
\\

& \emph{3} & 
0.7152 & 0.6532 & 0.7122 & 
0.5518 & 0.7467 & 0.5373 & 
0.6109 & 0.5728 & 0.5801 & 
0.6245 & 0.5937 & 0.6635 & 
0.5098 & 0.7508 & 0.6229 & 
0.6425 & 0.6964 & 0.4715
\\

\hline
\multirow{3}{*}{\emph{\textbf{F1}}} & 1 & 
0.5547 & 0.4598 & 0.5610 &
0.2926 & 0.6842 & 0.4268 &            
0.3285 & 0.3453 & 0.2850 &
0.2844 & 0.2428 & 0.3795 &
0.0796 & 0.0667 & 0.0656 &
0.6448 & 0.9248 & 0.4713
\\

& \emph{2} & 
0.5445 & 0.4672 & 0.5423 &
0.6059 & 0.6988 & 0.3646 &
0.3333 & 0.3573 & 0.2845 &
0.3703 & 0.2812 & 0.3956 &
0.0334 & 0.0754 & 0.1664 & 
0.6301 & 0.6914 & 0.4216 
\\

& \emph{3} &
0.5489 & 0.4606 & 0.5605 & 
0.2958 & 0.6691 & 0.3801 & 
0.3679 & 0.3158 & 0.3203 & 
0.3009 & 0.2719 & 0.3703 & 
0.0619 & 0.2033 & 0.0649 & 
0.6548 & 0.9372 & 0.4946
\\

\hline
\multirow{3}{*}{\emph{\textbf{MCC}}} & 1 & 
0.4295 & 0.3096 & 0.4577 &
0.0623 & 0.4799 & 0.0889 &
0.2285 & 0.1789 & 0.1425 &
0.1929 & 0.1647 & 0.3027 &
0.0481 & 0.0576 & 0.0497 &
0.1564 & 0.1543 & -0.0402 
\\

& \emph{2} &
0.4117 & 0.2809 & 0.4446 &
0.2992 & 0.4522 & 0.0298 &
0.2050 & 0.1376 & 0.1454 &
0.2943 & 0.2032 & 0.3449 &
0.0036 & 0.1113 & 0.1781 & 
0.1162 & 0.1648 & -0.0409
\\

& \emph{3} & 
0.4114 & 0.2872 & 0.4538 & 
0.0666 & 0.4554 & 0.0505 & 
0.2497 & 0.1342 & 0.1601 & 
0.2056 & 0.1726 & 0.2886 & 
0.0122 & 0.2086 & 0.0464 & 
0.1612 & 0.4101 & -0.0274 
\\

\hline
\multirow{3}{*}{\emph{\textbf{AUC}}} & 1 & 
0.7133 & 0.6563 & 0.7120 &
0.5485 & 0.7560 & 0.5549 &
0.5919 & 0.5947 & 0.5642 &
0.6009 & 0.5827 & 0.6682 &
0.5285 & 0.5222 & 0.5653 &
0.6373 & 0.5909 & 0.4608
\\

& \emph{2} & 
0.7182 & 0.6645 & 0.6998 &
0.6372 & 0.7313 & 0.5128 &
0.5909 & 0.5862 & 0.5726 &
0.6952 & 0.6609 & 0.7002 &
0.5267 & 0.7071 & 0.6771 &
0.5793 & 0.6682 & 0.4635
\\

& \emph{3} & 
0.7152 & 0.6532 & 0.7122 &
0.5518 & 0.7467 & 0.5373 &
0.6109 & 0.5728 & 0.5801 &
0.6245 & 0.5937 & 0.6635 &
0.4782* & 0.7062* & 0.5463* &
0.6425 & 0.6964 & 0.4715
\\

\hline
\end{longtable}
\end{tiny}
\end{center}
\end{landscape}
\begin{landscape}
\begin{center}
\begin{scriptsize}
\begin{longtable}{ | r  l | c | c | c | 
                            c | c | c | 
                            c | c | c | 
                            c | c | c | 
                            c | c | c | 
                            c | c | c | }
\caption{All Metrics}\label{tab:all-metrics2} \\

\hline
\textbf{\textit{Measure}}                            & 
\textbf{\textit{Metr.}}                              & 
\multicolumn{3}{| c |}{\textbf{\textit{synapse}}}    & 
\multicolumn{3}{| c |}{\textbf{\textit{xalan}}}      &
\multicolumn{3}{| c |}{\textbf{\textit{xerces}}}     &
\multicolumn{3}{| c |}{\textbf{\textit{hadoop 0.1}}} &
\multicolumn{3}{| c |}{\textbf{\textit{hadoop 0.2}}} &
\multicolumn{3}{| c |}{\textbf{\textit{hadoop 0.3}}} \\
\hline
\endfirsthead
\hline
\multicolumn{14}{c}{\tablename\ \thetable\ -- \textit{Continued from previous page}} \\
\hline

\\
\hline
\endhead
\hline
\textbf{\textit{Measure}}                            & 
\textbf{\textit{Metr.}}                              & 
\multicolumn{3}{| c |}{\textbf{\textit{synapse}}}    & 
\multicolumn{3}{| c |}{\textbf{\textit{xalan}}}      &
\multicolumn{3}{| c |}{\textbf{\textit{xerces}}}     &
\multicolumn{3}{| c |}{\textbf{\textit{hadoop 0.1}}} &
\multicolumn{3}{| c |}{\textbf{\textit{hadoop 0.2}}} &
\multicolumn{3}{| c |}{\textbf{\textit{hadoop 0.3}}} \\
\endfoot
\hline
\endlastfoot

\multirow{2}{*}{\emph{\textbf{Balance}}} & \emph{C1} & 
\multicolumn{3}{| c |}{0.9209} & 
\multicolumn{3}{| c |}{0.0944} & 
\multicolumn{3}{| c |}{0.8219} & 
\multicolumn{3}{| c |}{0.9381} &
\multicolumn{3}{| c |}{0.7600} &
\multicolumn{3}{| c |}{0.8132}
\\

& \emph{C2} & 
\multicolumn{3}{| c |}{0.1944} & 
\multicolumn{3}{| c |}{0.9755} & 
\multicolumn{3}{| c |}{0.3826} & 
\multicolumn{3}{| c |}{0.1559} & 
\multicolumn{3}{| c |}{0.4777} & 
\multicolumn{3}{| c |}{0.3970}
\\ 

\hline
\multirow{3}{*}{\emph{\textbf{Correlation}}} & \emph{C2} & 
\multicolumn{3}{| c |}{0.2310} &
\multicolumn{3}{| c |}{0.0663} &
\multicolumn{3}{| c |}{0.2839} &
\multicolumn{3}{| c |}{0.2868} &
\multicolumn{3}{| c |}{0.0520} &
\multicolumn{3}{| c |}{0.3171}
\\

& \emph{C3} & 
\multicolumn{3}{| c |}{0.7762} & 
\multicolumn{3}{| c |}{0.5275} &
\multicolumn{3}{| c |}{0.6338} &
\multicolumn{3}{| c |}{0.7741} &
\multicolumn{3}{| c |}{0.6006} &
\multicolumn{3}{| c |}{0.6330}
\\

& \emph{C4} & 
\multicolumn{3}{| c |}{0.9727} &
\multicolumn{3}{| c |}{0.0000} & 
\multicolumn{3}{| c |}{0.8878} &
\multicolumn{3}{| c |}{1.0000} &
\multicolumn{3}{| c |}{1.0000} &
\multicolumn{3}{| c |}{0.9810}
\\

\hline
\multirow{2}{*}{\emph{\textbf{Dimension.}}} & \emph{T2} & 
\multicolumn{3}{| c |}{0.0781} &
\multicolumn{3}{| c |}{0.0220} &
\multicolumn{3}{| c |}{0.0340} & 
\multicolumn{3}{| c |}{0.0496} & 
\multicolumn{3}{| c |}{0.0366} &
\multicolumn{3}{| c |}{0.0332}
\\

& \emph{T3} & 
\multicolumn{3}{| c |}{0.0078} & 
\multicolumn{3}{| c |}{0.0022} &
\multicolumn{3}{| c |}{0.0034} &
\multicolumn{3}{| c |}{0.0142} &
\multicolumn{3}{| c |}{0.0052} &
\multicolumn{3}{| c |}{0.0095}
\\

\hline
\multirow{3}{*}{\emph{\textbf{Linearity}}} & \emph{L1} & 
\multicolumn{3}{| c |}{0.3139} & 
\multicolumn{3}{| c |}{0.0285} & 
\multicolumn{3}{| c |}{0.3050} &
\multicolumn{3}{| c |}{0.3770} &
\multicolumn{3}{| c |}{0.3327} &
\multicolumn{3}{| c |}{0.2658}
\\

& \emph{L2} & 
\multicolumn{3}{| c |}{0.1534} &
\multicolumn{3}{| c |}{0.0117} &
\multicolumn{3}{| c |}{0.1342} &
\multicolumn{3}{| c |}{0.1870} &
\multicolumn{3}{| c |}{0.1664} & 
\multicolumn{3}{| c |}{0.1305}
\\

& \emph{L3} &
\multicolumn{3}{| c |}{0.1421} &
\multicolumn{3}{| c |}{0.0110} &
\multicolumn{3}{| c |}{0.1211} &
\multicolumn{3}{| c |}{0.1832} &
\multicolumn{3}{| c |}{0.1644} & 
\multicolumn{3}{| c |}{0.1055}
\\

\hline
\multirow{6}{*}{\emph{\textbf{Neighbor.}}} & \emph{N1} & 
\multicolumn{3}{| c |}{0.4023} &
\multicolumn{3}{| c |}{0.0396} &
\multicolumn{3}{| c |}{0.2789} &
\multicolumn{3}{| c |}{0.4965} &
\multicolumn{3}{| c |}{0.4712} &
\multicolumn{3}{| c |}{0.4028}
\\

& \emph{N2} & 
\multicolumn{3}{| c |}{0.3780} & 
\multicolumn{3}{| c |}{0.1615} & 
\multicolumn{3}{| c |}{0.2667} &
\multicolumn{3}{| c |}{0.3474} &
\multicolumn{3}{| c |}{0.4611} &
\multicolumn{3}{| c |}{0.3151}
\\

& \emph{N3} & 
\multicolumn{3}{| c |}{0.2422} &
\multicolumn{3}{| c |}{0.0121} &
\multicolumn{3}{| c |}{0.1173} &
\multicolumn{3}{| c |}{0.2553} & 
\multicolumn{3}{| c |}{0.3037} & 
\multicolumn{3}{| c |}{0.2417}
\\

& \emph{N4} & 
\multicolumn{3}{| c |}{0.0859} &
\multicolumn{3}{| c |}{0.0308} &
\multicolumn{3}{| c |}{0.0833} &
\multicolumn{3}{| c |}{0.2411} &
\multicolumn{3}{| c |}{0.2408} &
\multicolumn{3}{| c |}{0.1374}
\\

& \emph{T1} & 
\multicolumn{3}{| c |}{0.0072} &
\multicolumn{3}{| c |}{0.0294} & 
\multicolumn{3}{| c |}{0.0057} &
\multicolumn{3}{| c |}{0.0127} &
\multicolumn{3}{| c |}{0.0079} &
\multicolumn{3}{| c |}{0.0118}
\\

& \emph{LSC} & 
\multicolumn{3}{| c |}{0.9797} & 
\multicolumn{3}{| c |}{0.8329} & 
\multicolumn{3}{| c |}{0.9601} & 
\multicolumn{3}{| c |}{0.9755} & 
\multicolumn{3}{| c |}{0.9838} & 
\multicolumn{3}{| c |}{0.9686}
\\

\hline
\multirow{3}{*}{\emph{\textbf{Network}}} & \emph{Density} & 
\multicolumn{3}{| c |}{0.8805} &
\multicolumn{3}{| c |}{0.8165} & 
\multicolumn{3}{| c |}{0.8530} & 
\multicolumn{3}{| c |}{0.8678} & 
\multicolumn{3}{| c |}{0.8610} & 
\multicolumn{3}{| c |}{0.8478}
\\

& \emph{ClsCoef} & 
\multicolumn{3}{| c |}{0.3143} & 
\multicolumn{3}{| c |}{0.3460} & 
\multicolumn{3}{| c |}{0.3429} & 
\multicolumn{3}{| c |}{0.3965} & 
\multicolumn{3}{| c |}{0.4411} &
\multicolumn{3}{| c |}{0.4281}
\\

& \emph{Hubs} & 
\multicolumn{3}{| c |}{0.9300} & 
\multicolumn{3}{| c |}{0.8564} & 
\multicolumn{3}{| c |}{0.8499} & 
\multicolumn{3}{| c |}{0.8595} & 
\multicolumn{3}{| c |}{0.9252} & 
\multicolumn{3}{| c |}{0.9305}
\\

\hline
\multirow{5}{*}{\emph{\textbf{Overlap}}} & \emph{F1} & 
\multicolumn{3}{| c |}{0.9438} & 
\multicolumn{3}{| c |}{0.9973} & 
\multicolumn{3}{| c |}{0.9496} & 
\multicolumn{3}{| c |}{0.9129} & 
\multicolumn{3}{| c |}{0.9958} & 
\multicolumn{3}{| c |}{0.8365}
\\

& \emph{F1v} & 
\multicolumn{3}{| c |}{0.3313} &
\multicolumn{3}{| c |}{0.3274} &
\multicolumn{3}{| c |}{0.3118} &
\multicolumn{3}{| c |}{0.5101} &
\multicolumn{3}{| c |}{0.8483} & 
\multicolumn{3}{| c |}{0.3019}
\\

& \emph{F2} & 
\multicolumn{3}{| c |}{0.0065} & 
\multicolumn{3}{| c |}{0.0000} & 
\multicolumn{3}{| c |}{0.0000} & 
\multicolumn{3}{| c |}{0.1818} & 
\multicolumn{3}{| c |}{0.0508} &
\multicolumn{3}{| c |}{0.0135}
\\

& \emph{F3} & 
\multicolumn{3}{| c |}{0.9141} & 
\multicolumn{3}{| c |}{0.3399} & 
\multicolumn{3}{| c |}{0.8129} & 
\multicolumn{3}{| c |}{0.8865} & 
\multicolumn{3}{| c |}{0.9791} &
\multicolumn{3}{| c |}{0.9479}
\\

& \emph{F4} & 
\multicolumn{3}{| c |}{0.7227} & 
\multicolumn{3}{| c |}{0.0055} & 
\multicolumn{3}{| c |}{0.6990} & 
\multicolumn{3}{| c |}{0.6879} & 
\multicolumn{3}{| c |}{0.9215} & 
\multicolumn{3}{| c |}{0.8294}
\\

\hline
\multirow{4}{*}{\emph{\textbf{Smooth.}}} & S1 & 
\multicolumn{3}{| c |}{0.2824} & 
\multicolumn{3}{| c |}{0.0319} & 
\multicolumn{3}{| c |}{0.2010} & 
\multicolumn{3}{| c |}{0.3643} & 
\multicolumn{3}{| c |}{0.3421} &
\multicolumn{3}{| c |}{0.2952}
\\

& \emph{S2} & 
\multicolumn{3}{| c |}{0.1638} & 
\multicolumn{3}{| c |}{0.1538} & 
\multicolumn{3}{| c |}{0.1479} & 
\multicolumn{3}{| c |}{0.1465} & 
\multicolumn{3}{| c |}{0.0908} &
\multicolumn{3}{| c |}{0.1134}
\\

& \emph{S3} &
\multicolumn{3}{| c |}{0.2422} &
\multicolumn{3}{| c |}{0.0121} &
\multicolumn{3}{| c |}{0.1122} &
\multicolumn{3}{| c |}{0.2411} &
\multicolumn{3}{| c |}{0.3246} & 
\multicolumn{3}{| c |}{0.2417}
\\

& \emph{S4} & 
\multicolumn{3}{| c |}{0.1300} & 
\multicolumn{3}{| c |}{0.0209} & 
\multicolumn{3}{| c |}{0.1056} & 
\multicolumn{3}{| c |}{0.2003} & 
\multicolumn{3}{| c |}{0.1863} & 
\multicolumn{3}{| c |}{0.1552}
\\

\hline
& &
\textit{NB} & \textit{DT} & \textit{NC} &
\textit{NB} & \textit{DT} & \textit{NC} &
\textit{NB} & \textit{DT} & \textit{NC} &
\textit{NB} & \textit{DT} & \textit{NC} &
\textit{NB} & \textit{DT} & \textit{NC} &
\textit{NB} & \textit{DT} & \textit{NC} 
\\

\hline
\multirow{3}{*}{\emph{\textbf{Precision}}} & 1 & 
0.6317 & 0.5581 & 0.6431 &
0.9963 & 0.9944 & 0.9971 &
0.9339 & 0.9374 & 0.9283 &
0.6933 & 0.4782 & 0.5091 &
0.1357 & 0.0286 & 0.1097 &
0.6467 & 0.3671 & 0.5821 
\\

& \emph{2} & 
0.5499 & 0.4756 & 0.6472 &
0.9942 & 0.9973 & 1.0000 &
0.9200 & 0.9395 & 0.9385 &
0.6515 & 0.5028 & 0.5933 &
0.1335 & 0.1998 & 0.1852 & 
0.6384 & 0.3965 & 0.6784 
\\

& \emph{3} & 
0.6325 & 0.5645 & 0.6444 &
0.9952 & 0.9955 & 0.9971 & 
0.9322 & 0.9347 & 0.9283 & 
0.6933 & 0.5246 & 0.5091 & 
0.1324 & 0.1059 & 0.1403 & 
0.6133 & 0.3666 & 0.5738
\\

\hline
\multirow{3}{*}{\emph{\textbf{Recall}}} & 1 & 
0.4927 & 0.5641 & 0.4950 &
0.8721 & 0.9911 & 0.3930 &
0.5768 & 0.9250 & 0.3007 &
0.3086 & 0.6962 & 0.3276 &
0.1736 & 0.1000 & 0.3305 &
0.7038 & 0.8410 & 0.7295 
\\

& \emph{2} &
0.5353 & 0.5741 & 0.5309 &
0.9511 & 0.7158 & 0.4923 &
0.6346 & 0.8836 & 0.2577 &
0.4814 & 0.6905 & 0.4447 & 
0.2634 & 0.3915 & 0.4613 & 
0.5322 & 0.8189 & 0.4900 
\\

& \emph{3} & 
0.5618 & 0.6636 & 0.5342 & 
0.9032 & 0.9933 & 0.4019 & 
0.5931 & 0.9218 & 0.3007 & 
0.3276 & 0.7795 & 0.3276 & 
0.2471 & 0.1638 & 0.3442 & 
0.7133 & 0.8457 & 0.7391 
\\

\hline
\multirow{3}{*}{\emph{\textbf{Fall Out}}} & 1 & 
0.1613 & 0.2309 & 0.1467 &
0.2167 & 0.3667 & 0.0500 &
0.1200 & 0.1832 & 0.0669 &
0.0835 & 0.3650 & 0.1437 &
0.2528 & 0.3141 & 0.5494 &
0.0464 & 0.1936 & 0.0820 
\\

& \emph{2} & 
0.2225 & 0.3016 & 0.1399 &
0.3867 & 0.0800 & 0.0000 &
0.1576 & 0.1607 & 0.0498 & 
0.1468 & 0.3506 & 0.1628 & 
0.3558 & 0.4480 & 0.5418 & 
0.0996 & 0.4086 & 0.1129 
\\

& \emph{3} & 
0.1767 & 0.2634 & 0.1579 & 
0.3167 & 0.2667 & 0.0500 & 
0.1271 & 0.1880 & 0.0669 & 
0.0835 & 0.3553 & 0.1437 & 
0.5278 & 0.3680 & 0.5611 & 
0.0566 & 0.1841 & 0.0872 
\\

\hline
\multirow{3}{*}{\emph{\textbf{Balanced}}} & 1 & 
0.6657 & 0.6666 & 0.6742 &
0.8277 & 0.8122 & 0.6715 &
0.7284 & 0.8710 & 0.6169 &
0.6126 & 0.6656 & 0.5920 &
0.4604 & 0.3929 & 0.3906 &
0.7739 & 0.7642 & 0.7619 
\\

& \emph{2} & 
0.6564 & 0.6362 & 0.6955 &
0.7822 & 0.8179 & 0.7461 &
0.7385 & 0.8614 & 0.6039 & 
0.6673 & 0.6699 & 0.6410 & 
0.4538 & 0.4717 & 0.4598 & 
0.7163 & 0.7051 & 0.6885 
\\

& \emph{3} & 
0.6926 & 0.7001 & 0.6881 & 
0.7933 & 0.8633 & 0.6759 & 
0.7330 & 0.8669 & 0.6169 & 
0.6221 & 0.7121 & 0.5920 & 
0.3597 & 0.3979 & 0.3916 & 
0.7784 & 0.7737 & 0.7688 
\\

\hline
\multirow{3}{*}{\emph{\textbf{F1}}} & 1 & 
0.5450 & 0.5495 & 0.5506 &
0.9295 & 0.9927 & 0.5623 &
0.7126 & 0.9311 & 0.4534 &
0.4071 & 0.5033 & 0.3942 &
0.0702 & 0.0444 & 0.0969 &
0.6307 & 0.3834 & 0.5626 
\\

& \emph{2} &
0.5286 & 0.5124 & 0.5754 &
0.9717 & 0.8268 & 0.6487 &
0.7500 & 0.9103 & 0.3993 &
0.5389 & 0.5776 & 0.5015 & 
0.1642 & 0.2637 & 0.2389 & 
0.5738 & 0.5184 & 0.5567 
\\

& \emph{3} & 
0.5823 & 0.5955 & 0.5721 & 
0.9467 & 0.9944 & 0.5712 & 
0.7244 & 0.9282 & 0.4534 & 
0.4235 & 0.5523 & 0.3942 & 
0.0805 & 0.0812 & 0.1134 & 
0.6092 & 0.3915 & 0.5657 
\\

\hline
\multirow{3}{*}{\emph{\textbf{MCC}}} & 1 & 
0.3601 & 0.3322 & 0.3780 &
0.1999 & 0.4730 & 0.0693 &
0.3998 & 0.7314 & 0.2375 &
0.2969 & 0.2480 & 0.2093 &
-0.0751 & -0.1679 & -0.1940 &
0.4862 & 0.2945 & 0.4438 
\\

& \emph{2} & 
0.3131 & 0.2548 & 0.4089 &
0.3450 & 0.1551 & 0.0962 & 
0.4167 & 0.6855 & 0.2254 &
0.3677 & 0.3349 & 0.3051 & 
-0.0992 & -0.0367 & -0.0672 & 
0.4569 & 0.3571 & 0.4334  
\\

& \emph{3} & 
0.3971 & 0.3855 & 0.3983 & 
0.1871 & 0.6163 & 0.0708 & 
0.4073 & 0.7251 & 0.2375 & 
0.3108 & 0.3369 & 0.2093 & 
-0.2479 & -0.1659 & -0.1773 & 
0.4592 & 0.2994 & 0.4363
\\

\hline
\multirow{3}{*}{\emph{\textbf{AUC}}} & 1 & 
0.6657 & 0.6666 & 0.6742 &
0.8277 & 0.8122 & 0.6715 &
0.7284 & 0.8710 & 0.6169 &
0.6126 & 0.6656 & 0.5920 &
0.4604 & 0.3929 & 0.3906 &
0.8543* & 0.8540* & 0.8571*
\\

& \emph{2} & 
0.6564 & 0.6362 & 0.6955 &
0.7822 & 0.8179 & 0.7461 &
0.7385 & 0.8614 & 0.6039 &
0.6673 & 0.6699 & 0.6410 &
0.4538 & 0.4717 & 0.4598 &
0.7163 & 0.7051 & 0.6885
\\

& \emph{3} & 
0.6926 & 0.7001 & 0.6881 &
0.7933 & 0.8633 & 0.6759 & 
0.7330 & 0.8669 & 0.6169 &
0.6221 & 0.7121 & 0.5920 &
0.3597 & 0.3979 & 0.3916 &
0.8480* & 0.8599* & 0.8538*
\\

\hline
\end{longtable}
\end{scriptsize}
\end{center}
\end{landscape}
\begin{landscape}
\begin{center}
\begin{tiny}
\begin{longtable}{ | r  l | c | c | c | 
                            c | c | c | 
                            c | c | c | 
                            c | c | c | 
                            c | c | c | }
\caption{All Metrics}\label{tab:all-metrics3} \\

\hline
\textbf{\textit{Measure}}                            & 
\textbf{\textit{Metr.}}                              & 
\multicolumn{3}{| c |}{\textbf{\textit{hadoop 0.4}}} & 
\multicolumn{3}{| c |}{\textbf{\textit{hadoop 0.5}}} &
\multicolumn{3}{| c |}{\textbf{\textit{hadoop 0.6}}} &
\multicolumn{3}{| c |}{\textbf{\textit{hadoop 0.7}}} &
\multicolumn{3}{| c |}{\textbf{\textit{hadoop 0.8}}} \\
\hline
\endfirsthead
\hline
\multicolumn{14}{c}{\tablename\ \thetable\ -- \textit{Continued from previous page}} \\
\hline

\\
\hline
\endhead
\hline
\textbf{\textit{Measure}}                            & 
\textbf{\textit{Metr.}}                              & 
\multicolumn{3}{| c |}{\textbf{\textit{hadoop 0.4}}} & 
\multicolumn{3}{| c |}{\textbf{\textit{hadoop 0.5}}} &
\multicolumn{3}{| c |}{\textbf{\textit{hadoop 0.6}}} &
\multicolumn{3}{| c |}{\textbf{\textit{hadoop 0.7}}} &
\multicolumn{3}{| c |}{\textbf{\textit{hadoop 0.8}}} \\
\endfoot
\hline
\endlastfoot

\multirow{2}{*}{\emph{\textbf{Balance}}} & \emph{C1} & 
\multicolumn{3}{| c |}{0.7395} & 
\multicolumn{3}{| c |}{0.6589} & 
\multicolumn{3}{| c |}{0.5642} &
\multicolumn{3}{| c |}{0.7056} &
\multicolumn{3}{| c |}{0.3534}
\\

& \emph{C2} & 
\multicolumn{3}{| c |}{0.5062} &
\multicolumn{3}{| c |}{0.6056} &
\multicolumn{3}{| c |}{0.7015} & 
\multicolumn{3}{| c |}{0.5501} & 
\multicolumn{3}{| c |}{0.8579}
\\ 

\hline
\multirow{3}{*}{\emph{\textbf{Correlation}}} & \emph{C2} & 
\multicolumn{3}{| c |}{0.1899} & 
\multicolumn{3}{| c |}{0.1601} &
\multicolumn{3}{| c |}{0.1705} &
\multicolumn{3}{| c |}{0.1997} &
\multicolumn{3}{| c |}{0.1241}
\\

& \emph{C3} & 
\multicolumn{3}{| c |}{0.5956} &
\multicolumn{3}{| c |}{0.5589} &
\multicolumn{3}{| c |}{0.5397} &
\multicolumn{3}{| c |}{0.5737} &
\multicolumn{3}{| c |}{0.4917}
\\

& \emph{C4} & 
\multicolumn{3}{| c |}{1.0000} &
\multicolumn{3}{| c |}{1.0000} &
\multicolumn{3}{| c |}{0.7692} &
\multicolumn{3}{| c |}{0.9720} & 
\multicolumn{3}{| c |}{0.2125}
\\

\hline
\multirow{2}{*}{\emph{\textbf{Dimensional.}}} & \emph{T2} & 
\multicolumn{3}{| c |}{0.0348} & 
\multicolumn{3}{| c |}{0.0322} & 
\multicolumn{3}{| c |}{0.0299} & 
\multicolumn{3}{| c |}{0.0280} &
\multicolumn{3}{| c |}{0.0292}
\\

& \emph{T3} & 
\multicolumn{3}{| c |}{0.0050} &
\multicolumn{3}{| c |}{0.0046} & 
\multicolumn{3}{| c |}{0.0042} & 
\multicolumn{3}{| c |}{0.0040} & 
\multicolumn{3}{| c |}{0.0042}
\\

\hline
\multirow{3}{*}{\emph{\textbf{Linearity}}} & \emph{L1} & 
\multicolumn{3}{| c |}{0.2950} & 
\multicolumn{3}{| c |}{0.2620} & 
\multicolumn{3}{| c |}{0.1996} & 
\multicolumn{3}{| c |}{0.2654} & 
\multicolumn{3}{| c |}{0.1134}
\\

& \emph{L2} & 
\multicolumn{3}{| c |}{0.1473} &
\multicolumn{3}{| c |}{0.1310} &
\multicolumn{3}{| c |}{0.0981} &
\multicolumn{3}{| c |}{0.1327} & 
\multicolumn{3}{| c |}{0.0550}
\\

& \emph{L3} &
\multicolumn{3}{| c |}{0.1408} & 
\multicolumn{3}{| c |}{0.1274} & 
\multicolumn{3}{| c |}{0.0875} &
\multicolumn{3}{| c |}{0.1254} &
\multicolumn{3}{| c |}{0.0489}
\\

\hline
\multirow{6}{*}{\emph{\textbf{Neighborh.}}} & \emph{N1} & 
\multicolumn{3}{| c |}{0.4378} & 
\multicolumn{3}{| c |}{0.4009} &
\multicolumn{3}{| c |}{0.3333} &
\multicolumn{3}{| c |}{0.3840} & 
\multicolumn{3}{| c |}{0.1708}
\\

& \emph{N2} & 
\multicolumn{3}{| c |}{0.4207} &
\multicolumn{3}{| c |}{0.3960} &
\multicolumn{3}{| c |}{0.3457} &
\multicolumn{3}{| c |}{0.3682} &
\multicolumn{3}{| c |}{0.3151}
\\

& \emph{N3} & 
\multicolumn{3}{| c |}{0.3234} & 
\multicolumn{3}{| c |}{0.2857} &
\multicolumn{3}{| c |}{0.1838} &
\multicolumn{3}{| c |}{0.2200} &
\multicolumn{3}{| c |}{0.1208}
\\

& \emph{N4} & 
\multicolumn{3}{| c |}{0.1990} & 
\multicolumn{3}{| c |}{0.1429} & 
\multicolumn{3}{| c |}{0.1282} & 
\multicolumn{3}{| c |}{0.0960} & 
\multicolumn{3}{| c |}{0.0417}
\\

& \emph{T1} & 
\multicolumn{3}{| c |}{0.0101} &
\multicolumn{3}{| c |}{0.0096} &
\multicolumn{3}{| c |}{0.0100} & 
\multicolumn{3}{| c |}{0.0087} & 
\multicolumn{3}{| c |}{0.0143}
\\

& \emph{LSC} & 
\multicolumn{3}{| c |}{0.9708} &
\multicolumn{3}{| c |}{0.9725} &
\multicolumn{3}{| c |}{0.9686} &
\multicolumn{3}{| c |}{0.9786} &
\multicolumn{3}{| c |}{0.9428}
\\

\hline
\multirow{3}{*}{\emph{\textbf{Network}}} & \emph{Density} & 
\multicolumn{3}{| c |}{0.8433} & 
\multicolumn{3}{| c |}{0.8430} & 
\multicolumn{3}{| c |}{0.8232} &
\multicolumn{3}{| c |}{0.8426} & 
\multicolumn{3}{| c |}{0.8030}
\\

& \emph{ClsCoef} & 
\multicolumn{3}{| c |}{0.4453} & 
\multicolumn{3}{| c |}{0.4398} & 
\multicolumn{3}{| c |}{0.4383} &
\multicolumn{3}{| c |}{0.4435} &
\multicolumn{3}{| c |}{0.4465}
\\

& \emph{Hubs} & 
\multicolumn{3}{| c |}{0.9121} &
\multicolumn{3}{| c |}{0.9152} &
\multicolumn{3}{| c |}{0.8929} & 
\multicolumn{3}{| c |}{0.9074} &
\multicolumn{3}{| c |}{0.8912}
\\

\hline
\multirow{5}{*}{\emph{\textbf{Overlap}}} & \emph{F1} & 
\multicolumn{3}{| c |}{0.9641} & 
\multicolumn{3}{| c |}{0.9643} &
\multicolumn{3}{| c |}{0.9519} &
\multicolumn{3}{| c |}{0.9370} & 
\multicolumn{3}{| c |}{0.9711}
\\

& \emph{F1v} & 
\multicolumn{3}{| c |}{0.5797} &
\multicolumn{3}{| c |}{0.6427} & 
\multicolumn{3}{| c |}{0.4057} & 
\multicolumn{3}{| c |}{0.4816} & 
\multicolumn{3}{| c |}{0.3269}
\\

& \emph{F2} & 
\multicolumn{3}{| c |}{0.1107} & 
\multicolumn{3}{| c |}{0.3023} &
\multicolumn{3}{| c |}{0.0257} & 
\multicolumn{3}{| c |}{0.0552} & 
\multicolumn{3}{| c |}{0.0329}
\\

& \emph{F3} & 
\multicolumn{3}{| c |}{0.9900} &
\multicolumn{3}{| c |}{0.9032} &
\multicolumn{3}{| c |}{0.9530} &
\multicolumn{3}{| c |}{0.9600} &
\multicolumn{3}{| c |}{0.8750}
\\

& \emph{F4} & 
\multicolumn{3}{| c |}{0.9453} &
\multicolumn{3}{| c |}{0.8479} & 
\multicolumn{3}{| c |}{0.8846} & 
\multicolumn{3}{| c |}{0.9000} & 
\multicolumn{3}{| c |}{0.6875}
\\

\hline
\multirow{4}{*}{\emph{\textbf{Smooth.}}} & S1 & 
\multicolumn{3}{| c |}{0.3100} &
\multicolumn{3}{| c |}{0.2778} &
\multicolumn{3}{| c |}{0.2446} & 
\multicolumn{3}{| c |}{0.2731} & 
\multicolumn{3}{| c |}{0.1255}
\\

& \emph{S2} & 
\multicolumn{3}{| c |}{0.1075} &
\multicolumn{3}{| c |}{0.1108} &
\multicolumn{3}{| c |}{0.1122} & 
\multicolumn{3}{| c |}{0.1043} & 
\multicolumn{3}{| c |}{0.0961}
\\

& \emph{S3} &
\multicolumn{3}{| c |}{0.3284} &
\multicolumn{3}{| c |}{0.2857} & 
\multicolumn{3}{| c |}{0.1709} &
\multicolumn{3}{| c |}{0.2080} & 
\multicolumn{3}{| c |}{0.1208}
\\

& \emph{S4} & 
\multicolumn{3}{| c |}{0.1516} & 
\multicolumn{3}{| c |}{0.1646} & 
\multicolumn{3}{| c |}{0.1175} & 
\multicolumn{3}{| c |}{0.1631} & 
\multicolumn{3}{| c |}{0.0625}
\\

\hline
& &
\textit{NB} & \textit{DT} & \textit{NC} &
\textit{NB} & \textit{DT} & \textit{NC} &
\textit{NB} & \textit{DT} & \textit{NC} &
\textit{NB} & \textit{DT} & \textit{NC} &
\textit{NB} & \textit{DT} & \textit{NC} 
\\

\hline
\multirow{3}{*}{\emph{\textbf{Precision}}} & 1 &
0.4150 & 0.4632 & 0.3438 &
0.3619 & 0.1294 & 0.2726 &
0.3700 & 0.1600 & 0.0000 &
0.3149 & 0.3302 & 0.2639 &
0.2667 & 0.0000 & 0.0417 
\\

& \emph{2} & 
0.4267 & 0.2003 & 0.3316 & 
0.2332 & 0.1710 & 0.2274 & 
0.3267 & 0.2323 & 0.5909 & 
0.3433 & 0.2340 & 0.3688 & 
0.0719 & 0.1130 & 0.0384 
\\

& \emph{3} & 
0.3778 & 0.3571 & 0.3395 & 
0.2267 & 0.1775 & 0.2155 &
0.1567 & 0.2318 & 0.0000 & 
0.3067 & 0.2968 & 0.2693 & 
0.0569 & 0.1000 & 0.0417 
\\

\hline
\multirow{3}{*}{\emph{\textbf{Recall}}} & 1 & 
0.2478 & 0.4868 & 0.3240 &
0.2174 & 0.1687 & 0.3062 &
0.2192 & 0.0500 & 0.0000 &
0.2554 & 0.5484 & 0.3218 &
0.0750 & 0.0000 & 0.6000 
\\

& \emph{2} & 
0.4165 & 0.4451 & 0.3321 &
0.2713 & 0.5327 & 0.2982 & 
0.5695 & 0.7195 & 0.4889 & 
0.3500 & 0.5833 & 0.3643 & 
0.2667 & 0.5000 & 0.2667 
\\

& \emph{3} & 
0.3164 & 0.4344 & 0.3565 & 
0.1841 & 0.2228 & 0.3723 & 
0.2067 & 0.1525 & 0.0000 & 
0.2954 & 0.5537 & 0.4270 & 
0.3500 & 0.0750 & 0.6000 
\\

\hline
\multirow{3}{*}{\emph{\textbf{Fall Out}}} & 1 & 
0.0990 & 0.1244 & 0.1629 &
0.0885 & 0.2025 & 0.1959 &
0.0683 & 0.0524 & 0.0000 &
0.0928 & 0.1716 & 0.1774 &
0.0474 & 0.0272 & 0.6045 
\\

& \emph{2} & 
0.1498 & 0.4898 & 0.1637 &
0.2145 & 0.4981 & 0.1926 & 
0.1765 & 0.3742 & 0.0792 & 
0.1585 & 0.4321 & 0.1740 & 
0.2203 & 0.2902 & 0.4134 
\\

& \emph{3} & 
0.1522 & 0.1915 & 0.1757 & 
0.1387 & 0.2685 & 0.3335 & 
0.0820 & 0.1207 & 0.0000 & 
0.1181 & 0.2058 & 0.1989 & 
0.2789 & 0.1480 & 0.6000  
\\

\hline
\multirow{3}{*}{\emph{\textbf{Balanced}}} & 1 & 
0.5744 & 0.6812 & 0.5805 &
0.5645 & 0.4831 & 0.5551 &
0.5754 & 0.4988 & 0.5000 &
0.6753 & 0.7644 & 0.6602 &
0.5138 & 0.4864 & 0.4977 
\\

& \emph{2} & 
0.6334 & 0.4777 & 0.5842 & 
0.5284 & 0.5173 & 0.5528 & 
0.6965 & 0.6726 & 0.7048 & 
0.5958 & 0.5756 & 0.5951 & 
0.5232 & 0.6049 & 0.4266
\\

& \emph{3} & 
0.5821 & 0.6214 & 0.5904 & 
0.5227 & 0.4772 & 0.5194 & 
0.5623 & 0.5159 & 0.5000 & 
0.6827 & 0.7519 & 0.7021 & 
0.5356 & 0.4635 & 0.5000 
\\

\hline
\multirow{3}{*}{\emph{\textbf{F1}}} & 1 & 
0.2896 & 0.4405 & 0.3090 &
0.2507 & 0.1397 & 0.2680 &
0.1813 & 0.0762 & 0.0000 &
0.2347 & 0.2974 & 0.2314 &
0.1016 & 0.0000 & 0.0735 
\\

& \emph{2} & 
0.3952 & 0.2738 & 0.3268 & 
0.2268 & 0.2554 & 0.2453 & 
0.4051 & 0.3372 & 0.4850 & 
0.3403 & 0.3224 & 0.3525 & 
0.1111 & 0.1785 & 0.0644 
\\

& \emph{3} & 
0.3165 & 0.3699 & 0.3324 & 
0.1834 & 0.1789 & 0.2679 & 
0.1486 & 0.0750 & 0.0000 & 
0.2524 & 0.2784 & 0.2502 & 
0.0919 & 0.0833 & 0.0735 
\\

\hline
\multirow{3}{*}{\emph{\textbf{MCC}}} & 1 & 
0.1832 & 0.3324 & 0.1645 &
0.1589 & -0.0435 & 0.1108 & 
0.1595 & 0.0220 & 0.0000 &
0.1572 & 0.2647 & 0.1267 &
0.0932 & -0.0347 & -0.0088
\\

& \emph{2} & 
0.2686 & -0.0322 & 0.1628 & 
0.0600 & 0.0153 & 0.0882 & 
0.3136 & 0.2382 & 0.4386 & 
0.1930 & 0.1153 & 0.2090 & 
0.0182 & 0.1147 & -0.0751
\\

& \emph{3} & 
0.1767 & 0.2116 & 0.1781 & 
0.0489 & -0.0172 & 0.0347 & 
0.1053 & 0.0395 & 0.0000 & 
0.1599 & 0.2277 & 0.1595 & 
0.0078 & 0.0055 & 0.0000 
\\

\hline
\multirow{3}{*}{\emph{\textbf{AUC}}} & 1 & 
0.5744 & 0.6812 & 0.5805 &
0.5645 & 0.4831 & 0.5551 &
0.5754 & 0.4988 & 0.5000 &
0.6091* & 0.7656* & 0.6052* &
0.5138 & 0.4864 & 0.4977
\\

& \emph{2} & 
0.6334 & 0.4777 & 0.5842 & 
0.5284 & 0.5173 & 0.5528 &
0.6965 & 0.6726 & 0.7048 & 
0.5958 & 0.5756 & 0.5951 & 
0.5232 & 0.6049 & 0.4266
\\

& \emph{3} & 
0.5821 & 0.6214 & 0.5904 &
0.5227 & 0.4772 & 0.5194 & 
0.5623 & 0.5159 & 0.5000 &
0.6173* & 0.7449* & 0.6576* &
0.5356 & 0.4635 & 0.5000
\\

\hline
\end{longtable}
\end{tiny}
\end{center}
\end{landscape}

