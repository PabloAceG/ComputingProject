%%%%%%%%%%%%%%%%%%%%%%%%%%%%%%%%%%%%%%%%%%%%%%%%%%%%%%%%%%%%%%%%%%%%%%%%%%%
%
% Generic template for TFC/TFM/TFG/Tesis
%
% $Id: abstract.tex,v 1.8 2014/04/17 17:28:45 macias Exp $
%
% By:
%  + Javier Mac�as-Guarasa. 
%    Departamento de Electr�nica
%    Universidad de Alcal�
%  + Roberto Barra-Chicote. 
%    Departamento de Ingenier�a Electr�nica
%    Universidad Polit�cnica de Madrid   
% 
% Based on original sources by Roberto Barra, Manuel Oca�a, Jes�s Nuevo,
% Pedro Revenga, Fernando Herr�nz and Noelia Hern�ndez. Thanks a lot to
% all of them, and to the many anonymous contributors found (thanks to
% google) that provided help in setting all this up.
%
% See also the additionalContributors.txt file to check the name of
% additional contributors to this work.
%
% If you think you can add pieces of relevant/useful examples,
% improvements, please contact us at (macias@depeca.uah.es)
%
% Copyleft 2013
%
%%%%%%%%%%%%%%%%%%%%%%%%%%%%%%%%%%%%%%%%%%%%%%%%%%%%%%%%%%%%%%%%%%%%%%%%%%%

\chapter*{Abstract}
\label{cha:abstract}

\addcontentsline{toc}{chapter}{Abstract}

When developing code, machine learning techniques can help in a variety of ways, 
one of them is predicting defective modules. There are many works trying to 
identify defective modules but in this work we focus on on of the problems 
related to the datasets used to induce such modules, its complexity, including 
imbalance. Data mining classifiers tend to optimize the performance by focusing 
on the majority class first. That means that they perform poorly when dealing 
with imbalanced datasets, where the minority class is outnumbered by the number 
of samples of the majority class; or when samples overlap. In real world 
problems, this kind of distribution is common. This project tries to analyze 
those metrics to see what complexity metrics affect the classification
performance.

\textbf{Keywords:} \mybookkeywords.

%%% Local Variables:
%%% TeX-master: "../book"
%%% End:


