%%%%%%%%%%%%%%%%%%%%%%%%%%%%%%%%%%%%%%%%%%%%%%%%%%%%%%%%%%%%%%%%%%%%%%%%%%%
%
% Generic template for TFC/TFM/TFG/Tesis
%
% $Id: resumen.tex,v 1.8 2014/04/17 17:28:45 macias Exp $
%
% By:
%  + Javier Mac�as-Guarasa. 
%    Departamento de Electr�nica
%    Universidad de Alcal�
%  + Roberto Barra-Chicote. 
%    Departamento de Ingenier�a Electr�nica
%    Universidad Polit�cnica de Madrid   
% 
% Based on original sources by Roberto Barra, Manuel Oca�a, Jes�s Nuevo,
% Pedro Revenga, Fernando Herr�nz and Noelia Hern�ndez. Thanks a lot to
% all of them, and to the many anonymous contributors found (thanks to
% google) that provided help in setting all this up.
%
% See also the additionalContributors.txt file to check the name of
% additional contributors to this work.
%
% If you think you can add pieces of relevant/useful examples,
% improvements, please contact us at (macias@depeca.uah.es)
%
% Copyleft 2013
%
%%%%%%%%%%%%%%%%%%%%%%%%%%%%%%%%%%%%%%%%%%%%%%%%%%%%%%%%%%%%%%%%%%%%%%%%%%%

\chapter*{Resumen}
\label{cha:resumen}
\markboth{Resumen}{Resumen}

\addcontentsline{toc}{chapter}{Resumen}

En el campo de la miner�a de datos, los clasificadores tratan de optimizar su 
rencimiento centr�ndose sobre la clase mayoritaria. Esto implica que un 
clasificador no dar� los resultados esperados en conjuntos de datos 
desbalanceados, donde las muestras de la clase minoritaria son insignificantes 
en comparaci�n con las muestras de la clase mayoritaria; o cuando dichas 
muestras se solapan. En problemas reales, este tipo de distribuci�n de datos es 
com�n. Esta t�sis intenta analizar las m�tricas anteriormente mencionadas (y 
otras) para ver como estas afectan al rendimiento del clasificador. Adem�s, la
tesis prueba t�cnicas que deber�an mejorar el rendimiento de los clasificadores,
medido mediante m�tricas anal�ticas sacadas a partir de la matriz de confusi�n.

\textbf{Palabras clave:} \mybookpalabrasclave.

%%% Local Variables:
%%% TeX-master: "../book"
%%% End:


