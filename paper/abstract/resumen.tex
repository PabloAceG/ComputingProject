%%%%%%%%%%%%%%%%%%%%%%%%%%%%%%%%%%%%%%%%%%%%%%%%%%%%%%%%%%%%%%%%%%%%%%%%%%%
%
% Generic template for TFC/TFM/TFG/Tesis
%
% $Id: resumen.tex,v 1.8 2014/04/17 17:28:45 macias Exp $
%
% By:
%  + Javier Mac�as-Guarasa. 
%    Departamento de Electr�nica
%    Universidad de Alcal�
%  + Roberto Barra-Chicote. 
%    Departamento de Ingenier�a Electr�nica
%    Universidad Polit�cnica de Madrid   
% 
% Based on original sources by Roberto Barra, Manuel Oca�a, Jes�s Nuevo,
% Pedro Revenga, Fernando Herr�nz and Noelia Hern�ndez. Thanks a lot to
% all of them, and to the many anonymous contributors found (thanks to
% google) that provided help in setting all this up.
%
% See also the additionalContributors.txt file to check the name of
% additional contributors to this work.
%
% If you think you can add pieces of relevant/useful examples,
% improvements, please contact us at (macias@depeca.uah.es)
%
% Copyleft 2013
%
%%%%%%%%%%%%%%%%%%%%%%%%%%%%%%%%%%%%%%%%%%%%%%%%%%%%%%%%%%%%%%%%%%%%%%%%%%%

\chapter*{Resumen}
\label{cha:resumen}
\markboth{Resumen}{Resumen}

\addcontentsline{toc}{chapter}{Resumen}

En el �rea de desarrollo de c�digo, el \textit{Machine Learning} puede ayudar
de varias maneras, como por ejemplo en la detecci�n de c�digo defectuoso. 
Existen diferentes estudios que tienen como objetivo el identificar dichos 
m�dulos defectuosos, aunque en este trabajo, nos centramos en los problemas 
relacionados con sets de datos creados a partir de dichos m�dulos, su 
complejidad, incluido el desbalanceo. En el campo de la miner�a de datos, los 
clasificadores tratan de optimizar su rendimiento centr�ndose sobre la clase 
mayoritaria. Esto implica que un clasificador no dar� los resultados esperados 
en conjuntos de datos desbalanceados, donde las muestras de la clase minoritaria 
son insignificantes en comparaci�n con las muestras de la clase mayoritaria; o 
cuando dichas muestras se solapan. Este proyecto intenta analizar las m�tricas 
anteriormente mencionadas para ver como estas afectan al rendimiento del 
clasificador. 

\textbf{Palabras clave:} \mybookpalabrasclave.

%%% Local Variables:
%%% TeX-master: "../book"
%%% End:


